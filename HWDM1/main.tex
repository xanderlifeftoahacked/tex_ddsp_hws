\documentclass{article}
\usepackage{ucs} 
\usepackage[utf8x]{inputenc} % Включаем поддержку UTF8  
\usepackage[russian]{babel}  % Включаем пакет для поддержки русского языка  
\usepackage{amsmath}
\usepackage{tikz}
\usepackage{setspace}
\usepackage{amsfonts}
\usepackage{geometry}
\usepackage{quoting}
\usepackage{booktabs}
\usepackage[a4paper, left=0.5cm, right=0.5cm, top=0cm, bottom=0cm]{geometry}


\begin{document}
\setlength{\parindent}{0pt}
\begin{Large}
    \textsf{\textbf{Дискра ДЗ №1}}
    
    Шамаев Александр    
\end{Large}
\vspace{1cm}

\textsf{\textbf{1.}}
\begin{quote}
\leftskip=0.2cm    
(a) $p \lor \neg q$ \quad
(b) $q \land r \lor \neg(q \lor r)$ \quad
(c) $\neg (p \lor q \lor r) \lor (\neg p \land \neg q \land r) \lor (p \land q \land \neg r)$ \\
\end{quote}

\textsf{\textbf{2.}}
\begin{quote}
\leftskip=0.2cm    
1. Если ровно одно из чисел $x, y, z$ четное, значит 
их сумма: 

чет. + (нечет. + нечет.) = чет. + чет. = чет., т.е. $w$ - четное

2. Если все числа $x, y, z$ четные, значит $w$ четное как сумма четных.

Получается, что если одно из двух условий выполняется, то $w$ - четное.

Иными словами, $A \equiv (B \lor C)$
\end{quote}
\textsf{\textbf{3.}}
\begin{quote}
\leftskip=0.2cm    
(a) Построим таблицу истинности:

\hspace{0.5cm}
\begin{tabular}{ m m m m }
    p & q & r & F(p, q, r) \\
\hline
    0 & 0 & 0 & 0 \\
    0 & 0 & 1 & 1 \\
    0 & 1 & 0 & 0 \\
    0 & 1 & 1 & 1 \\
    1 & 0 & 0 & 1 \\
    1 & 0 & 1 & 1 \\
    1 & 1 & 0 & 1 \\
    1 & 1 & 1 & 1 \\
\end{tabular}

Ответ: истинно всегда, кроме ($p = q = r = 0$) и ($p = r = 0, q = 1$) \\

(b) $p_1 \rightarrow \underbrace{(p_2 \rightarrow \underbrace{(p_3 \rightarrow \underbrace{(p_4 \rightarrow p_5)}_{(3)}))}_{(2)}}_{(1)}$\\

\hspace{0.5cm}
Ложно только в случае, если $p_1 = 1, \ (1) = 0$.

\hspace{0.5cm}
(1) = 0 только в случае, если $p_2 = 1, \ (2) = 0$

\hspace{0.5cm}
(2) = 0 только в случае, если $p_3 = 1, \ (3) = 0$

\hspace{0.5cm}
(3) = 0 только в случае, если $p_4 = 1, \ p_5 = 0$

Ответ: истинно всегда, кроме ($p_1 = p_2 = p_3 = p_4 = 1, \ p_5 = 0$).\\

(c) Подставим $p = 1$: 

\begin{center}
    
$(1 \land (q \rightarrow r)) \leftrightarrow 
((1 \land q) \rightarrow (1 \land r))$

$(q \rightarrow r) \leftrightarrow 
(q \rightarrow r)$
\end{center}

\hspace{0.5cm}
Является тавтологией. Подставим $p = 0$:
\begin{center}
    
$(0 \land (q \rightarrow r)) \leftrightarrow 
((0 \land q) \rightarrow (0 \land r))$

$(0 \leftrightarrow (0 \rightarrow 0))$

$(0 \leftrightarrow 1)$
\end{center}

\hspace{0.5cm}
Высказывание ложно.

 Ответ: истино при $p = 1$\\

(d) Составим таблицу истинности:

\hspace{0.5cm}
\begin{tabular}{ m m m m }
    p & q & r & F(p, q, r) \\
\hline
0 & 0 & 0 & 1 \\
0 & 0 & 1 & 1 \\
0 & 1 & 0 & 1 \\
0 & 1 & 1 & 1 \\
1 & 0 & 0 & 1 \\
1 & 0 & 1 & 1 \\
1 & 1 & 0 & 1 \\
1 & 1 & 1 & 1 \\
\end{tabular}

Ответ: является тавтологией \\

\end{quote}

\textsf{\textbf{4.}}
\begin{quote}
\leftskip=0.2cm    
1. Пары могут пересекаться (Например, ЛРЛЛРЛ: 4 ответа "да" \ , \ 4 пары, каждый рыцарь входит в две пары)

2. Правильным ответом будет 66 (решение ниже)

Решение: если всего будет 100 лжецов, ответов "да" \ не будет. Если добавим рыцаря - ответов "да" \ 
будет два. Если добавить рыцаря рядом с имеющимся рыцарем, ответов "да" \ останется столько же.
Если поставить после лжеца (ЛРЛРЛ), то ответов "да" \ будет на два больше. И т. д. Мы максимизируем количество ответов "да" \ , поэтому будем ставить рыцарей таким образом.
Получается, $N_{yes} = 2 * N_{knights}, $

$N_{liars} = 100 - N_{knights} = 100 - N_{yes}/2, \ N_{liars} = N_{yes} = x$

$x = 100 - \frac{x}{2} \Longleftrightarrow x = 66 \frac{2}{3}$

Округляем вниз, т.к. кол-во лжецов - натуральное число. Получается лжецов было 66. Расположение получится, например, таким: ...ЛРРЛ ЛРЛРЛРЛРЛРЛРЛ ... ...ЛЛЛЛ...

\end{quote}
\end{document}