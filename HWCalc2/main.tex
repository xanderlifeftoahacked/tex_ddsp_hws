\documentclass{article}
\usepackage{ucs} 
\usepackage[utf8x]{inputenc} % Включаем поддержку UTF8  
\usepackage{amsmath,amssymb}
\usepackage{tikz}
\usepackage{setspace}
\usepackage{amsfonts}
\usepackage{geometry}
\usepackage{quoting}
\usepackage[russian]{babel}  % Включаем пакет для поддержки русского языка  
\usepackage[a4paper, left=0.5cm, right=0.5cm, top=0cm, bottom=0cm]{geometry}
\begin{document}
\setlength{\parindent}{0pt}
\begin{Large}
    \textsf{\textbf{МатАн ДЗ №2}}
    
    Шамаев Александр    
\end{Large}
\vspace{1cm}


\textsf{\textbf{(1)}}

\begin{quote}
   Пусть $\{x_n\}$ является ограниченной, т.е.:
   $
   \exists C_0 \forall n: |x_{n}| < C_0 \Leftrightarrow \exists C_0 \forall n: -C_0 < x_{n} < C_0
   $, иными словами: 
   \begin{cases}
   $C_1 = C_0, \ \forall n: x_n < C_1$ - т.е. послед. ограничена сверху\\ 
   $C_2 = -C_0, \ \forall n: x_n > C_2$ - т.е. послед ограничена снизу \\ 
   \end{cases}
   
   Теперь докажем в обратную сторону. Пусть система ограничена сверху и снизу:
   \begin{cases}
   $\exists C_1, \ \forall n: x_n < C_1$ \\ 
   $\exists C_2, \ \forall n: x_n > C_2$ \\ 
   \end{cases} \Longrightarrow $ \ C_0 = max(|C_1|, |C_2|), \ \forall n: x_n < |C_0|$. 
   
   Т.е. послед. ограничена. ЧТД.
   
\end{quote}


\textsf{\textbf{(2)}}

\begin{quote}
    
(a) $\{a , a , a ,  ...\}$

(b) $a_n = \frac{1}{n} + a$

(c) Если $n$ - четное: $a_n = a$, если нечетное: $a_n = \frac{1}{n} + a$

(d) $\{0, a, 0, a, 0, a, 0, ...\}$

\end{quote}

\textsf{\textbf{(3)}}

\begin{quote}
Если $x$ - предел \{$x_n$\}, значит (по определению) в любой окрестности $x$ содержатся почти все члены $\{x_n\}$, т.е. все элементы за исключением
их конечного числа. Если мы добавим в начало этой последовательности конечное число элементов, то в любой окрестности $x$ по-прежнему будут содержатся почти все элементы, значит $x$ по-прежнему будет пределом \{$x_n$\}. Аналогично с отбрасыванием конечного числа элементов.  
\end{quote}

\textsf{\textbf{(4)}}

\begin{quote}
Найдем такое $C$, что $\forall n \in \mathbb{N}: |\frac{1 - n}{\sqrt{n^2 + 1}}| < C$

Т.к. $n$ - натуральное: $1 - n \leq 0, \ \sqrt{n^2 + 1} > 0$, т.е. подмодульное выр. всегда $\leq 0$, 

Тогда: $\frac{n - 1}{\sqrt{n^2 + 1}} < C \Leftrightarrow 
\frac{n^2 - 2n + 1}{n^2 + 1} < C^2 \Leftrightarrow n^2 - 2n + 1 - C^2 n^2 - C^2 < 0 \Leftrightarrow$

$\Leftrightarrow n^2(1 - C^2) - 2n + (1 - C^2) < 0$

Слева уравнение параболы. Нерав-о верно для $\forall n$, когда $f(x_{B}) < 0, \ (1 - C^2) < 0$:
$
\begin{cases}
   \frac{2^2}{2^2(1 - C^2)^2}(1 - C^2) - 2\frac{2}{2(1 - C^2)} + (1 - C^2) < 0 \\
   C^2 > 1
\end{cases} \Longleftrightarrow
\begin{cases}
   \frac{1}{(1 - C^2)} - \frac{2}{(1 - C^2)} + (1 - C^2) < 0 \\
   |C| > 1
\end{cases}
$ 

Рассмотрим первое неравенство ($t = (1 - C^2), \ t < 0$):
$\frac{1}{t} - \frac{2}{t} + t < 0 \Leftrightarrow t^2 - 1 > 0 \Leftrightarrow$

$\Leftrightarrow (t - 1)(t + 1) > 0 \Leftrightarrow (1 - C^2 - 1)(1 - C^2 + 1) > 0 \Leftrightarrow -C^2
(2 - C^2) > 0 \Leftrightarrow C^2 > 2$

Т.е. нашу последовательность можно ограничить $C > \sqrt{2}$, например, $C = 2$. ЧТД

\end{quote}


\textsf{\textbf{(5)}}
\begin{quote}
   Если последовательность неограничена, то $\forall C \exists n: |a_n| \geq C$ 
   
   $a_n = \begin{cases}  \frac{1}{n} , n \ \vdots \ 2\\ n , n \ \not \vdots \ 2 \end{cases}$

   Подпоследовательность с нечетными индексами неограничена (всегда можно взять $n = C + 1$ или $n = C + 2$). Значит 
   вся последовательность неограничена. ЧТД.
\end{quote}


\textsf{\textbf{(6)}}
\begin{quote}
(a) 
\begin{quote}
   Из определения предела:
   $\forall \varepsilon > 0 \ \exists N \ \forall n > N \ : \ |\frac{1}{\sqrt{3n - 11}} - 0| < \varepsilon$
   
    $\sqrt{3n - 11} > 0$ при натуральных $n$ $\Leftrightarrow $ дробь всегда положительна, тогда:
    
    $\frac{1}{\sqrt{3n - 11}} < \varepsilon \Longleftrightarrow
    3n - 11 > \frac{1}{\varepsilon ^ 2} \Longleftrightarrow n > \frac{1}{3 \varepsilon ^ 2} + \frac{11}{3}$

    Неравенство будет выполнятся для $\forall n > N$ при $N = \left\lceil{\frac{1}{3 \varepsilon ^ 2} + \frac{11}{3}}\right\rceil + 1$. Сходимость доказана.
\end{quote}

(b) 
\begin{quote}
   Из определения предела:
   $\forall \varepsilon > 0 \ \exists N \ \forall n > N \ : \ |\frac{2n + 3}{n ^ 2} - 0| < \varepsilon$
   
    $n^2 > 0, \ 2n + 3 > 0$ при натуральных $n$, т.е. дробь всегда положительна, тогда:

    $\frac{2n + 3}{n^2} < \varepsilon \Longleftrightarrow - \varepsilon n^2 + 2n + 3 < 0
    \Longleftrightarrow \varepsilon n^2 - 2n - 3 > 0 \Longleftrightarrow $
    
    $\Longleftrightarrow (n - \frac{1 + \sqrt{1 + 3 \varepsilon}}{\varepsilon})(n - \frac{1 - \sqrt{1 + 3 \varepsilon}}{\varepsilon}) > 0
    $

    Неравенство будет выполнятся для $\forall n > N$ при $N = \left\lceil{\frac{1 + \sqrt{1 + 3 \varepsilon}}{\varepsilon}}\right\rceil + 1$ (берем больший корень, потому что при $n$, находящемся между корнями, неравенство не выполняется) . Сходимость доказана.
\end{quote}

(c) 
\begin{quote}
   Из определения предела:
   $\forall \varepsilon > 0 \ \exists N \ \forall n > N \ : \ |\frac{\cos n}{\sqrt{n}} - 0| < \varepsilon$

   
   Косинус принимает значения от -1 до 1, значит дробь можем ограничить сверху так:
   $
    |\frac{\cos n}{\sqrt{n}}| \leq \frac{1}{\sqrt{n}}
   $

  $\frac{1}{\sqrt{n}} < \varepsilon \Longleftrightarrow n > \frac{1}{\varepsilon ^ 2}$ 
  
    Неравенство будет выполнятся для $\forall n > N$ при $N = \left\lceil{\frac{1}{\varepsilon ^ 2}}\right\rceil + 1$. Сходимость доказана.
\end{quote}
   
\end{quote}

\end{document}
