\documentclass{article}
\usepackage{ucs} 
\usepackage[utf8x]{inputenc} % Включаем поддержку UTF8  
\usepackage[russian]{babel}  % Включаем пакет для поддержки русского языка  
\usepackage{amsmath, amssymb}
\usepackage{tikz}
\usepackage{setspace}
\usepackage{amsfonts}
\usepackage{geometry}
\usepackage{quoting}
\usepackage{centernot}
\usepackage{booktabs}
\usepackage[a4paper, left=0.5cm, right=0.5cm, top=0cm, bottom=0cm]{geometry}

\let\emptyset\varnothing

\begin{document}
\setlength{\parindent}{0pt}
\begin{Large}
    \textsf{\textbf{Дискра ДЗ №3}}
    
    Шамаев Александр    
\end{Large}
\vspace{1cm}



\textsf{\textbf{1.}}
\begin{quote}
\begin{quote}
    a) Всего четных шестизначных чисел: A = $9 \cdot 10 \cdot 10 \cdot 10 \cdot 10 \cdot 5$

    Не содержащих "7": B = $8 \cdot 9 \cdot 9 \cdot 9 \cdot 9 \cdot 5$

    Чисел, содержащих хотя бы одну "7": A - B =  \fbox{187560}
\end{quote}

\begin{quote}

    b) Всего четных шестизначных чисел: A = $9 \cdot 10 \cdot 10 \cdot 10 \cdot 10 \cdot 5$

    Не содержащих "8": B = $8 \cdot 9 \cdot 9 \cdot 9 \cdot 9 \cdot 4$

    Чисел, содержащих хотя бы одну "8": A - B = \fbox{240048}
\end{quote}
\end{quote}

\textsf{\textbf{2.}}
\begin{quote}
    На каждую позицию в слове у нас есть 3 вида буквы. 
    
    Всего слов длины $n$ будет A $ = 3^n$. 
    
    Всего слов из одного вида букв будет B = 3.
    
    Слов, содержащих ровно два вида букв будет C $ =(2^n - 2) \cdot C_{3}^{2} = 3 \cdot 2^n - 6$ 
    
    (вычитаем два варианта, когда все буквы одинаковы, умножаем на кол-во способов выбрать два вида букв)

    Тогда слов, содержащих все три буквы будет $A - B - C = \fbox{3^n - 3 \cdot 2^n + 3}$
    
\end{quote}

\textsf{\textbf{3.}}
\begin{quote}
    $10^6 = 100^3 = 1000^2$

    В последовательности содержится 10 шестых степеней ($1^6, 2^6 , ... , 10^6)$ - кубов и квадратов одновременно, 
    100 кубов, 1000 квадратов. Четвертые степени являются квадратами, отдельно считать их не нужно. 
    
    Тогда искомых чисел будет (вычитаем 10, т.к. посчитали шестые степени два раза, можно сказать по принципу включений/исключений):
    
    $10^6 - (100 + 1000 - 10) = \fbox{998910}$
\end{quote}

\textsf{\textbf{4.}}
\begin{quote}
  Чисел, кратных трем будет $\frac{33000}{3} = 11000 \quad (1 \cdot 3 , 2 \cdot 3, ... , 11000 \cdot 3)$
  
  Кратных пяти - $\frac{33000}{5} = 6600$
  
  Кратных 11 - $\frac{33000}{11} = 3000$

  Чисел, кратных пяти и трем - $\frac{33000}{15} = 2200$
  
  Чисел, кратных пяти и 11 - $\frac{33000}{55} = 600$
  
  Чисел, кратных трем и 11 - $\frac{33000}{33} = 1000$
  
  Чисел, кратных трем, пяти и 11 - $\frac{33000}{165} = 200$

  Тогда искомых чисел будет (можно сказать, по принципу включений/исключений)
  
  $33000 - 11000 - 6600 - 3000 + 2200 + 600 + 1000 - 200 = \fbox{16000}$
\end{quote}
\end{document}