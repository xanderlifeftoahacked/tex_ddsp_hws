\documentclass{article}
\usepackage{ucs} 
\usepackage[utf8x]{inputenc} % Включаем поддержку UTF8  
\usepackage{amsmath}
\usepackage{tikz}
\usepackage{setspace}
\usepackage{amsfonts}
\usepackage{geometry}
\usepackage{quoting}
\usepackage[russian]{babel}  % Включаем пакет для поддержки русского языка  
\usepackage[a4paper, left=0.5cm, right=0.5cm, top=0cm, bottom=0cm]{geometry}
\begin{document}
\setlength{\parindent}{0pt}
\begin{Large}
    \textsf{\textbf{ЛинАл ДЗ №4}}
    
    Шамаев Александр    
\end{Large}
\vspace{1cm}

\textsf{\textbf{(1)}}

\begin{quote}
Нет. Мы можем домножить любой $V \in \textbf{W}$. На скаляр $-1 \in \mathbb{R}$. 
И получить $V^{'} \not \in \textbf{W}$. Значит $\textbf{W}$ не явл. подпространством.
\end{quote}

\textsf{\textbf{(2)}}

\begin{quote}
Нет. Возьмем $V_1 = \begin{pmatrix} 1 \\ 3 \end{pmatrix}, \ V_2 = \begin{pmatrix} -2 \\ -1 \end{pmatrix}$. Они лежат в \textbf{U}. Однако $V_1 + V_2 = \begin{pmatrix} -1 \\ 2\end{pmatrix}$ не лежит.
\end{quote}



\textsf{\textbf{(3)}}

\begin{quote}
$a = \begin{pmatrix} 0 \\ 1 \end{pmatrix}  , \ b = \begin{pmatrix}
    1 \\ 0
\end{pmatrix}, \ a + b = \begin{pmatrix}
   1 \\ 1 
\end{pmatrix} \Rightarrow
a + b \not \in D$

$c = \begin{pmatrix}
    1 \\ 0
\end{pmatrix}, \ \lambda = 2, \ c \lambda \not \in D$
\end{quote}

\textsf{\textbf{(4)}}
\begin{quote}
(a) $\lambda_1 x^2 + \lambda_2 x^2 = (\lambda_1 + \lambda_2) x^2 = \lambda_{new} x^2$ - мн. замкнуто относительно сложения


\hspace{0.5cm} $\lambda_2 (\lambda_1 x^2) = \lambda_2 \lambda_1 x^2 = (\lambda_1 \lambda_2) x^2 = \lambda_{new} x^2$ - мн. замкнуто относительно

\hspace{0.5cm} умножения на скаляр

\hspace{0.5cm} $\lambda_0 = 0: \ \lambda_0 x^2 + \lambda x^2 = \lambda x^2$ - нулевой элемент присутствует

\hspace{0.5cm} Или можно просто сказать, что множество является векторной оболочкой $(Span_{\mathbb{R}}\{x^2\})$. 

\hspace{0.5cm} \boxed{\text{Множество является подпространством.}}

(b) В множестве отсутствует нулевой полином. $\lambda_0 + x^2 + \lambda_1 + x^2  = \lambda_{new} + \textbf{2}x^2 \not = \lambda_1 + x^2$ 

\hspace{0.5cm} при любых $\lambda_0$.

\hspace{0.5cm} \boxed{\text{Множество не является подпространством.}}

(c) Множество $W$ не замкнуто относительно умножения на скаляр:

\hspace{0.5cm} Возьмем полином $p(x) = x^3$ и скаляр $\lambda = 0.1$. $\lambda p(x) \not \in W$

\hspace{0.5cm} \boxed{\text{Множество не является подпространством.}}

(d) (все коэф. $p_0$ равны 0) $p_0(x) + p_2(x) = p_2(x)$ - нулевой элемент присутствует

\hspace{0.5cm} $p_3(x) = p_1(x) + p_2(x)$ лежит в $W$: $p_3(0) = (p_1 + p_2)(0) = p_1(0) + p_2(0) = 0$

\hspace{0.5cm} $\lambda p(x)$ лежит в W: $(\lambda p)(0) = \lambda p(0) = \lambda 0 = 0$

\hspace{0.5cm} \boxed{\text{Множество является подпространством.}}\\
\end{quote}



\textbf{\textsf{(5)}}
\begin{quote}
    $u = \begin{pmatrix} 1 \\ 3 \\ 2 \end{pmatrix}$, 
    $v = \begin{pmatrix} 2 \\ 0 \\ -1 \end{pmatrix}$
\end{quote}

\textbf{\textsf{(6)}}
\begin{quote}
$v_3 = 2 v_2$, значит любую линейную комбинацию $v_1, v_2, v_3$ можно представить в виде 
линейной комбинации $v_1, v_2$. Т.е. $Span_{\mathbb{R}}\{v_1, v_2, v_3\} = Span_{\mathbb{R}}\{v_1, v_2 \}$

$w = 1 \cdot v_1 + 1 \cdot v_2 \Longrightarrow w \in Span_{\mathbb{R}}\{v_1, v_2 \} 
\Leftrightarrow \boxed{w \in Span_{\mathbb{R}}\{v_1, v_2, v_3 \}}$

Проверим, представим ли $u$ в виде линейной комбинации $v_1, v_2$:

$w = \alpha_1 v_1 + \alpha_2 v_2 \Leftrightarrow \begin{cases} 1 \cdot \alpha_1 + 2 \cdot \alpha_2 = 8 \quad (1) \\
    0 \cdot \alpha_1 + 1 \cdot \alpha_2 = 4 \quad (2)\\
    -1 \cdot \alpha_1 + 3 \cdot \alpha_2 = 7 \quad (3)\\
\end{cases}$ 

$(2) \Leftrightarrow \alpha_2 = 4$, подставим в (1): $\alpha_1 = 8 - 2 \alpha_2 = 0$, подставим в (3): $-1 \cdot 0 + 3 \cdot 4 = 12 \not = 7$. 

Система не имеет решений. Значит $u \not \in Span_{\mathbb{R}}\{v_1, v_2 \} \Leftrightarrow \boxed {u \not \in Span_{\mathbb{R}}\{v_1, v_2 , v_3\}}$ 




\end{quote}

\end{document}