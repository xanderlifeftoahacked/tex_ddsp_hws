\documentclass{article}
\usepackage{ucs} 
\usepackage[utf8x]{inputenc} % Включаем поддержку UTF8  
\usepackage[russian]{babel}  % Включаем пакет для поддержки русского языка  
\usepackage{amsmath, amssymb}
\usepackage{tikz}
\usepackage{setspace}
\usepackage{amsfonts}
\usepackage{geometry}
\usepackage{quoting}
\usepackage{centernot}
\usepackage{booktabs}
\usepackage[a4paper, left=0.5cm, right=0.5cm, top=0cm, bottom=0cm]{geometry}

\let\emptyset\varnothing

\begin{document}
\setlength{\parindent}{0pt}
\begin{Large}
    \textsf{\textbf{Дискра ДЗ №4}}
    
    Шамаев Александр    
\end{Large}
\vspace{1cm}



\textsf{\textbf{1.}}
\begin{quote}
$2^{10}$ имеет 11 делителей: $1,  2  , ... , 2^{10}$

$3^{5}$ имеет 6 делителей: $1,  3  , ..., 3^{5}$

$5^{3}$ имеет 4 делителя: $1,  5  , ..., 5^{3}$

Получается, что 'составить' делитель числа $2^{10} \cdot 3^5 \cdot 5^3$ мы можем $11 \cdot 6 \cdot 4 = 264$ различными способами.
\boxed{\text{Ответ: 264}}
\end{quote}

\textsf{\textbf{2.}}
\begin{quote}
$a = 2k, \ k \centernot\vdots 2$, т.к. $a \centernot \vdots 4$

Все делители $k$ - нечетные. Назовем их $d_1, d_2,...,d_n$.

Тогда все нечетные делители $a$ - это $d_1, d_2, ... d_n$

А все четные  - это $2d_1, 2d_2, ... 2d_n$. Т.е. $a$ имеет $n$ четных и $n$ нечетных делителей. ЧТД.
\end{quote}

\textsf{\textbf{3.}}
\begin{quote}
$(p^2 - 1) = (p + 1)(p - 1)$

Если $p$ - простое и больше трех, то оно нечетное. Значит $(p + 1), (p - 1)$ - четные, причем идущие подряд. Значит одно из них точно кратно четырем (*). 

Т.е. их произведение кратно восьми. 

$p$ не м.б. кратно трем, т.к. простое.

Пусть $p \mod 3 \equiv 1$. Значит $p - 1 \mod 3 \equiv 0$. Т.е. $(p - 1) \ \vdots \ 3$

Пусть $p \mod 3 \equiv 2$. Значит $p + 1 \mod 3 \equiv 0$. Т.е. $(p + 1) \ \vdots \ 3$

Получаем, что произведение чисел кратно 8 и одно из чисел кратно 3. Значит произведение кратно 24. ЧТД. \\

Док-во (*): 

($p + 1$ и $p - 1$ - четные)

Предположим, 
$(p + 1) \ \centernot\vdots 4 \Longleftrightarrow$ $\frac{p + 1}{2} \mod 2 \equiv 1
\Longleftrightarrow \frac{p + 1}{2} - 1 \mod 2 \equiv 0 \Longleftrightarrow \frac{p - 1}{2} \mod 2 \equiv 0$

Иными словами, если $p + 1$, не кратно четырем, то $p - 1$ - кратно.

Аналогично доказывается, если $(p - 1) \ \centernot\vdots 4$.
\end{quote}

\textbf{\textsf{4.}}
\begin{quote}
НОД числителя и знаменателя равен одному: $(n^2 - n + 1, n ^ 2 + 1) = (n^2 - n + 1, n) = (1, n)$ 
, т.е. общих делителей, кроме единицы, у числ. и знам. нет, т.е.  дробь несократима. ЧТД.
\end{quote}

\textbf{\textsf{5.}}
\begin{quote}
$(19, 22) = (19, 3) = (1, 3) = 1$

$1 = 19 - 6 \cdot 3 =  19 - 6 \cdot (22 - 19) = 7 \cdot 19 - 6 \cdot 22$


\begin{center}

$19x + 22y = -21(7 \cdot 19 - 6 \cdot 22)$

$19x + 22y = 126 \cdot 22 - 147 \cdot 19$

$19(x + 147) = -22 (y - 126)$


$\begin{cases}
    x + 147 = -22k \\
    y - 126 = 19k \\
k \in \mathbb{Z}
\end{cases}\Leftrightarrow$
$\begin{cases}
    x = -22k - 147 \\
    y  = 19k + 126 \\
k \in \mathbb{Z}
\end{cases}$
\end{center}
\boxed{\text{Ответ: $x = -22k - 147, y = 19k + 126$ ($k \in \mathbb{Z}$)}}
\end{quote}

\textbf{\textsf{6.}}
\begin{quote}
$(b, c) = 1 \Leftrightarrow bx + cy = 1 \quad (x, y \in \mathbb{Z})$

$ad(bx + cy) = ad \quad (a, d \in \mathbb{Z})$

$ac(dy) + b(xad) = ad$

$ac(dy) + 2b(xad) = ad + b(xad)$

$ac(dy) + 2b(xad) = ad + b(xad)$

$\underbrace{ac(dy) + b(2xad)}_{=(ac, b)} = \underbrace{a(d) + b(xad)}_{=(a, b)}$

$(b, c) = 1$ \Longleftrightarrow
$\begin{cases}
    ac(dy) + b(2xad) = u \\
    a(d) + b(xad) = u 
\end{cases}$


ЧТД.
\end{quote}

\textbf{\textsf{7.}}
\begin{quote}
    Докажем методом мат.индукции.

    База: 8 = 5 + 3, т.е. 8 тугриков представить можно.
    
    Пусть сумму = k можно представить монетами по 3 и по 5 тугриков.
    
    Докажем, что можно представить сумму = k + 1:

    
    Если у нас есть хотя бы одна монета в 5 тугриков, то мы можем заменить монету в 5 тугриков на две монеты по 3 тугрика и получить сумму = k + 1.

    Если монеты в 5 тугриков нет, то у нас есть хотя бы три монеты в 3 тугрика, т.к. меньшим числом монет по 3 тугрика мы не сможем собрать сумму $\geq 8$ монет (по условию задачи, нам надо минимум 8). 
    
    
    Мы можем заменить их на две монеты по 5 тугриков и получить сумму = k + 1.

    Переход доказан. ЧТД.

\end{quote}

\end{document}