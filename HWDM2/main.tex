\documentclass{article}
\usepackage{ucs} 
\usepackage[utf8x]{inputenc} % Включаем поддержку UTF8  
\usepackage[russian]{babel}  % Включаем пакет для поддержки русского языка  
\usepackage{amsmath, amssymb}
\usepackage{tikz}
\usepackage{setspace}
\usepackage{amsfonts}
\usepackage{geometry}
\usepackage{quoting}
\usepackage{booktabs}
\usepackage[a4paper, left=0.5cm, right=0.5cm, top=0cm, bottom=0cm]{geometry}

\let\emptyset\varnothing

\begin{document}
\setlength{\parindent}{0pt}
\begin{Large}
    \textsf{\textbf{Дискра ДЗ №2}}
    
    Шамаев Александр    
\end{Large}
\vspace{1cm}



\textsf{\textbf{1.}}
\begin{quote}


\begin{center}
$0 \cdot n + 1 \cdot (n - 1) + 2 \cdot (n - 2) + ... +
(n - 1) \cdot 1 + n \cdot 0 = \frac{(n-1)n(n+1)}{6}$ (*)

$(\sum_{i = 0}^{n} i \cdot (n - i) = \frac{(n-1)n(n+1)}{6})$ 
\end{center}

1. База ($n = 1$): $ 0 * 1 + 1(1-1) = 0 = \frac{0}{6}$ - верно

2. Пусть утверждение (*) верно для $n=k$ : 

\begin{center}
$0 \cdot k + 1 \cdot (k - 1) + 2 \cdot (k - 2) + ... +
(k - 1) \cdot 1 + k \cdot 0 = \frac{(k-1)k(k+1)}{6}$
\end{center}

3. Проверим, верно ли для $n = k + 1$:

\begin{center}

$0 \cdot (k + 1)+ 1 \cdot (k + 1 - 1) + 2 \cdot (k + 1 - 2) + ... +
(k + 1 - 1) \cdot 1 + (k + 1) \cdot 0 = $

$= 1 \cdot k + 2 \cdot (k - 1) + ... + (k - 1) \cdot 2 +
k \cdot 1 = $


$ = 1 \cdot (k - 1) + 2 \cdot (k - 2) + ... +
(k - 1) \cdot 1 + (k + (k - 1) + ... + 2 + 1) = $

$= \frac{(k-1)k(k+1)}{6} + \frac{k(k + 1)}{2} = \frac{(k - 1)k(k+1) + 3k(k+1)}{6}=$
$ \frac{k(k+1)((k-1) + 3)}{6} = $

$= \frac{k(k+1)(k + 2)}{6}$ 
\end{center}

Является выражением из п.2, но с $k = k + 1$. Исходное равенство доказано.
\end{quote}

\textsf{\textbf{2.}}
\begin{quote}

1. База ($n$ (число городов) $=2$): из какого-то города можно попасть в любой другой (во второй) - верно

2. Пусть утверждение верно для $n=k$

3. Докажем верность для $n = k + 1$:

Пусть старый город, из которого можно было добраться в любой другой будет $A_x$, добавленный город - $A_y$.

Если $A_x$ и $A_y$ соединены дорогой из $A_x$ в $A_y$, то $A_x$ - по-прежнему город, откуда можно добраться в любой другой.

Если $A_x$ и $A_y$ соединены дорогой из $A_y$ в $A_x$, то рассмотрим два случая:

1. Все дороги, связанные с $A_y$ идут из $A_y$. Тогда из $A_y$ можно добраться в любой другой город.

2. Не все дороги, связанные с $A_y$ идут из $A_y$. Тогда в $A_y$ идет дорога из какого-то $A_z$. По предположению индукции из $A_x$ можно добраться в т.ч. в $A_z$. Значит можно сделать так: $A_x \rightarrow A_z \rightarrow A_y$, т.е. $A_x$ - по-прежнему город, из которого можно добраться в любой другой.

Утверждение доказано.
\end{quote}


\textsf{\textbf{3.}} ($x$ - произвольный элемент)
\begin{quote}
(a) 
\begin{quote}
    Пусть $\quad a = x \in A \ , \quad b = x \in B \ , \quad c = x \in C$ 

    Тогда правая часть равенства: $b \land \neg c$
    
    Перепишем левую часть равенства:

    \begin{center}
        $(a \lor b) \land \neg (a \lor c) \equiv (a \lor b) \land \neg a \land \neg c \equiv 
        ((a \land \neg a) \lor (b \land \neg a)) \land \neg c \equiv$
        
       $ 
        \equiv (b \land \neg a) \land \neg c \not \equiv b \land \neg c
        $
    \end{center}  

    Контрпример: $(a = 1, \ b = 1, \ c = 0)$
    
    Т.е. утверждение верно не для любых A, B, C. Ответ.

   
\end{quote}

(b) 
\begin{quote}
    Пусть $\quad a = x \in A \ , \quad b = x \in B \ , \quad c = x \in C , \ \quad d = x \in D$ 

    Левая часть:
    $((a \land \neg b) \land \neg c) \land \neg d  \equiv a \land \neg b \land \neg c \land \neg d$

    Правая часть:
    $a \land \neg (b \land \neg (c \land \neg d))$

    Правая часть не эквивалентна левой, контрпример:
    
    $(a = 1, \ b = 0, \  c = 1, \  d = 0)$ Ответ.

\end{quote}

(c) 
\begin{quote}
    Пусть $\quad a = x \in A \ , \quad b = x \in B \ , \quad c = x \in C $ 

    Левая часть:
    $a \land \neg (b \lor c) \equiv a \land \neg b \land \neg c$

    Правая часть:
    $(a \land \neg b) \land \neg c \equiv a \land \neg b \land \neg c$

    Левая и правая части эквивалентны.
    
    Ответ: верно
\end{quote}


(d) 
\begin{quote}
    Пусть $\quad a = x \in A \ , \quad b = x \in B $ 

    Левая часть: $(a \lor b) \land \neg (a \land \neg b) \equiv (a \lor b) \land (\neg a \lor b) \equiv b$ - 
    истинно только при $b = 1$

    Т.е. произвольный элемент лежит в $(A \cup B) \setminus (A \setminus B)$, тогда и только тогда, когда он лежит в $B$. Значит $(A \cup B) \setminus (A \setminus B) \subseteq B$. 
    
    Ответ: верно

\end{quote}
\end{quote}

\textsf{\textbf{4.}} ($x$ - произвольный элемент)
\begin{quote}
Пусть  $a_i = x \in A_i, \ 1 \leq i \leq n \ , \quad b_j = x \in B_j, \ 1 \leq j \leq n$:

Правая часть равенства: $(a_1 \land \neg b_1) \land (a_2 \land \neg b_2) \land ... \land (a_n \land \neg b_n)$

Левая часть равенства: 
\begin{center}
    
$(a_1 \land ... \land a_n) \land \neg (b_1 \lor ... \lor b_n) \equiv
a_1 \land ... \land a_n \land \neg b_1 \land ... \land \neg b_n \equiv$

$\equiv (a_1 \land \neg b_1) \land (a_2 \land \neg b_2) \land ... \land (a_n \land \neg b_n) $ 
ЧТД.
\end{center}
\end{quote}

\textsf{\textbf{5.}} ($x$ - произвольный элемент)
\begin{quote}
    Пусть $\quad a = x \in A \ , \quad b = x \in B \ , \quad c = x \in C $.
    Тогда выражение принимает вид:

    \begin{center}
    $a \lor b \Rightarrow c \land \neg (a \land b)$

    \end{center}

    Если оно истинно, то:
    

\left[ 
\begin{gathered} 
$a \lor b = 0$ (*) \\
    \begin{cases}
     $a \lor b = 1$ \\ 
     $c \land \neg (a \land b) = 1$ (**)
    \end{cases}
\end{gathered} 
\right

\end{quote}

\begin{quote}

(*) Истинно, только если произвольный элемент не может лежать в $A \cup B$,

т.е. $A \cup B = \emptyset$, следовательно $A \cap B = \emptyset$

(**) Истинно, только если $c$ - истинно и $(a \land b)$ - ложно, т.е. произвольный элемент не может лежать в $A \cap B$, иными словами $A \cap B = \emptyset $

Ответ: верно
\end{quote}

\end{document}