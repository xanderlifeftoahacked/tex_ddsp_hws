\documentclass{article}
\usepackage{ucs} 
\usepackage[utf8x]{inputenc} % Включаем поддержку UTF8  
\usepackage{amsmath}
\usepackage{tikz}
\usepackage{amssymb}
\usepackage{setspace}
\usepackage{amsfonts}
\usepackage{geometry}
\usepackage{quoting}
\usepackage[russian]{babel}  % Включаем пакет для поддержки русского языка  
\usepackage[a4paper, left=0.5cm, right=0.5cm, top=0cm, bottom=0cm]{geometry}
\begin{document}
\setlength{\parindent}{0pt}
\begin{Large}
    \textsf{\textbf{Коллоквиум 1 ЛинАЛ}}
    
    Я перепил вчера, но не был накурен в подъезде!
\end{Large}
\vspace{1cm}

\large\textsf{\textbf{Вопросы}}
\\
\\
(1) Что такое линейная система? 
\begin{quote}
    Конечный набор линейных уравнений.
\end{quote}

(2) Что такое решение линейной системы? 
\begin{quote}
    Набор чисел, при подстоновки которого ($x_i$ - переменная в системе, $x_1 \rightarrow s_1, \ x_2 \rightarrow s_2 , ... , \ x_n \rightarrow s_n$) в систему мы получаем тожденственные равенства.
\end{quote}

(3) Как понимать геометрически решение линейной системы? 
\begin{quote}
   Решения линенйного уравнения $ax_1 + bx_2 = c$ можно геометрически понимать как прямую на плоскости, где введена прямоугольная система координат $x_1Ox_2$. Значит решениями системы могут быть: точка, прямая, пустое множество.
\end{quote}

(4) Сколько решений может иметь линейная система? 
\begin{quote}
    Одно, ноль либо бесконечность
\end{quote}

(5) Что такое матричное обозначение линейной системы? 
\begin{quote}
Ну матрица че
\end{quote}

(6) Что такое элементарные преобразования? 
\begin{quote}
    Элементарными преобразованиями строк в произвольной матрице мы называем следующие операции: 
    
    1. Прибавить к любой строке матрице другую умноженную на проивзольное число. 
    
    2. Поменять местами строки. 
    
    3. Умножить строку матрицы на произовольное число. 
\end{quote}

(7) Что такое расширенная матрица линейной системы? 
\begin{quote}
Расширенной матрицей системы $\widetilde{A}=(A \mid B)$ называется матрица, полученная из матрицы коэффицентов системы $A$ , дописыванием справа после вертикальной черты столбца свободных членов.
\end{quote}

(8) Что такое матрица коэффициентов линейной системы? 
\begin{quote}
        Это матрица, в которой на $j$-ой строчке в $i$-ом столбце стоит коэффицент перед $x_i$ в $j$-ом уравнении системы
\end{quote}

(9) Что значит фраза “матрицы эквиваленты”? 
\begin{quote}
   Если матрица $M'$ получается из матрицы $M$ конечной комбинацией элементарных преобразований.
\end{quote}

(10) Что такое ведущий элементы строки матрицы? 
\begin{quote}
    Пусть имеем некоторую матрицу, мы говорим что её строка ненулевая если в этой строке есть хотя бы один ненулевой элемент. Аналогично мы называем ненулевые столбцы. Самый левый ненулевой элемент в ненулевой строке называется ведущим элементом этой строки.
\end{quote}

(11) Что такое ступенчатый вид матрицы? 
\begin{quote}
   Матрица $A$ называется ступенчатой, если:
   
   все ее нулевые строки стоят после ненулевых; 
   
   в каждой ненулевой строке, начиная со второго, ее главный элемент стоит правее (в столбце с большим номером) главного элемента предыдущей строки. 
\end{quote}

(12) Что такое приведённый (улучшенный) ступенчатый вид матрицы? 
\begin{quote}
Это ступенчатая матрица, у которой каждый ведущий элемент ненулевой строки - это единица, и он является единственным ненулевым элементом в своём столбце. 
\end{quote}

(13) Что такое метод Гаусса для решения линейных систем? 
\begin{quote}
    Это метод последовательного исключения переменных, когда с помощью элементарных преобразований система уравнений приводится к равносильной системе треугольного вида, из которой последовательно, начиная с последних (по номеру), находятся все переменные системы 
\end{quote}

(14) Что такое векторное пространство над $\mathbb{R}$?
\begin{quote}
   Векторное пространство это множество $V\ne \varnothing$ на котором есть две операции
\[
 \oplus : V \times V \to V, \qquad \odot : \mathbb{R} \times V \to V,
\]
для которых верны следующие условия

 \begin{enumerate}
        \item $\m{a} \oplus (\m{b} \oplus \m{c}) = (\m{a} \oplus \m{b}) \oplus \m{c}$, для любых $\m{a},\m{b}, \m{c} \in \m{V}$ (ассоциативность)
        \item $\m{a} \oplus \m{b} = \m{b}\oplus \m{a}$, для любых $\m{a}, \m{b} \in \m{V}$ (коммутативность)
        \item Существует такой вектор, называемый \textit{нулевым вектором}, $\m{o} \in \m{V}$, что $\m{o}\oplus \m{a} = \m{a}$, для любого $\m{a} \in \m{V}$ (существование нуля)
        \item Для любого вектора $\m{a} \in \m{V}$ существует такой вектор $-\m{a}\in \m{V}$, что $\m{a} \oplus (-\m{a})  = \m{o}$ (сущестование противополжоных)
        \item $(\alpha+ \beta) \odot \m{a} = \alpha \odot \m{a} \oplus \beta \odot \m{a}$ для любых чисел $\alpha, \beta \in \mathbb{R}$ и любого вектора $\m{a} \in \m{V},$ здесь $\alpha + \beta$ это обычное сложение чисел в $\mathbb{R}$ (дистрибутивность умножения вектора на скаляр относительно сложения скаляров) 
        \item $(\alpha \cdot \beta) \odot \m{a} = \alpha\odot(\beta \odot \m{a})$ для любых чисел $\alpha, \beta \in \mathbb{R}$ и любого вектора $\m{a} \in \m{V}$, здесь $\alpha \cdot \beta$ -- обычное умножение чисел в $\mathbb{R}$ (ассоциативность умножения на скаляр)
        \item $\alpha\odot (\m{a} \oplus \m{b}) = \alpha \odot \m{a} \oplus \alpha \odot \m{b}$ для любых векторов $\m{a},\m{b} \in \m{V}$ и любого числа $\alpha \in \mathbb{R}$ (дистрибутивность умножения вектора на скаляр относительно сложения векторов)
        \item $1 \odot \m{a} = \m{a}$ для любого вектора $\m{a} \in \m{V}$ (существование единицы)
    \end{enumerate}


 
\end{quote}

(15) Что такое векторное подпространство векторного пространства $\mathbb{R}^n$?
\begin{quote}
    $W$ - векторное подпространство $V$, если:
    
    1. $ W \subseteq V, W \ne \varnothing$
    
    2. $\mathbf{o} \in W$

    3. $\forall a, b \in W: \ a + b \in W$
    
    4. $\forall a \in W \ \forall \alpha \in W: \ \alpha a \in W$
\end{quote}

(16) Что такое линейная комбинация векторов? 
\begin{quote}
   $V$ - векторное пространство над $F$, $v_1, v_2 ... , v_n \in V$, $a_1, a_2 ... , a_n \in F$. Тогда линейной комбинацией $v_1 ... v_n$ будет: $a_1 v_1 + a_2 v_2 + ... + a_n v_n$
\end{quote}

(17) Что такое линейная оболочка?
\begin{quote}
   Если даны $a_1  ...  a_n \in V$, то ЛО, натянутой на $a_1 ... a_n$ называется мн-во $Span_\mathbb{R} (a1, ... , a_n)$ - множество всех линейных комб. векторов $a_1 ... a_n$
\end{quote}

(18) Что такое линейно независимое подмножество в векторном пространстве? 
\begin{quote}
    Система векторов ${v_1,...,v_k}$ из векторного пространства называется линейно зависимой (ЛЗ), если существует нетривиальная линейная комбинация этих векторов, равная нулю. Иначе система называется линейно независимой (ЛНЗ). 
\end{quote}

(19) Что такое базис векторного пространства? 
\begin{quote}
    $V$ - векторное пр-во, $a_1 ... a_n$ - ЛНЗ и $Span_\mathbb{R} = V$.
    Тогда ${a_1 , ... , a_n}$ - базис $V$, $\dim V = n$ (размерность $V$)
\end{quote}

(20) Что такое координаты вектора? 
\begin{quote}
   Пусть $E = \{e_1, . . . , e_n \}$– некоторое конечное множество векторов в некотором векторном пространстве. Пусть $W = Span_\mathbb{R} (e_1, . . . , e_n)$ – векторное пространство натянутое
на множество $E$. Тогда это значит, что для любого вектора $w \in W$ можно найти такие
числа $\alpha_1, . . . , \alpha_n \in \mathbb{R}$ что
$w = \alpha_1 e_1 + ... + \alpha_n e_n$.
Эти числа называются координатами вектора $w$ в базисе $E$.
\end{quote}

(21) Что такое $\mathbb{R}^n$? 
\begin{quote}
    $\mathbb{R}^n = \underbrace{\mathbb{R} \times ... \times \mathbb{R}}_{\text{$n$ раз}} $ = $\{ \begin{pmatrix} x_1 \\ \vdots \\ x_n\end{pmatrix} x_1 ... x_n \in \mathbb{R}\}$
\end{quote}

(22) Что такое изоморфизм векторных пространств? 
\begin{quote}
Биективное отображение $\varphi: V \rightarrow W$ называется изоморфизмом, если $\forall a, b \in V$ и $\forall \lambda \in \mathbb{R}$:

$
\varphi(a + b) = \varphi(a) + \varphi(b) 
$

$
\varphi(\lambda a) = \lambda \varphi(a)
$
\end{quote}
(23) Что такое линейное отображение векторных пространств?
\begin{quote}
    Отображение между множествами $\varphi: V \rightarrow W$ называется линейным, если $\forall a, b \in V$ и $\forall \lambda \in \mathbb{R}$:

$
\varphi(a + b) = \varphi(a) + \varphi(b) 
$

$
\varphi(\lambda a) = \lambda \varphi(a)
$
    
\end{quote}
\\
\\
\\
\large\textsf{\textbf{Теоремы}}
\\
\\
(1) Если расширенные матрицы двух линейных систем эквивалентны, то эти систе- мы имеют одно и то же множество решений. Докажите. 
\begin{quote}
    Пусть дана линейная система:

    $
    \begin{cases}
        L_1(x) = b_1 \\
        \dots \\
        L_i(x) = b_i \\ 
        \dots \\
        L_j(x) = b_j \\ 
        \dots \\
        L_m(x) = b_m \\ 
    \end{cases}
    $
    
    Рассмотрим первое эл. преобразование, т.е. прибавление к одной строке другой, умноженной на число:
    
    $
    \begin{cases}
        L_1(x) = b_1 \\
        \dots \\
        L_i(x) + \lambda \cdot L_j(x) = b_i + \lambda \cdot b_j \\ 
        \dots \\
        L_j(x) = b_j \\ 
        \dots \\
        L_m(x) = b_m \\ 
    \end{cases}
    $

    Пусть $s = (s_1, ..., s_n)$ - решение первоначальной системы. Тогда имеем тождества $L_1(s) \equiv b_1$ и т.д..

    Если вместо $x$ подставим набор $s$, то все уравнения, кроме $i$-го превратятся в такие же тождества, что и в первоначальной. Рассмотрим $i$-ое уравнение: $L'_i(s) := L_i(s) + \lambda \cdot L_j(s) = b_i + \lambda \cdot b_j =: b'_i$ т.е., мы получили тождество $L'_i(s) \equiv b'_i$, что показывает, что $s$ - тоже решение для новой системы.

    Докажем теперь, что любое решение новой системы будет и решением первоначальной: к $i$-ому уравнению прибавим $j$-ое, умноженное на $-\lambda$. Тогда мы получим первоначальное систему. А сохранение множества решений при таком преобразовании мы уже доказали. Доказательство теоремы для остальных преобразований аналогично и более простое. Гойда гол.

    
\end{quote}

(2) Любую матрицу с помощью элементарных преобразований можно привести к ступенчатому виду, а также к редуцированному ступенчатому виду. Докажите. 
\begin{quote}
    Пусть дана матрица: $A = 
    \begin{pmatrix} 
    a_{11} & a_{12} & \dots & a_{1n} \\
    a_{21} & a_{22} & \dots & a_{2n} \\ 
    \vdots & \vdots & \ddots & \vdots \\
    a_{m1} & a_{m2} & \dots & a_{mn} \\ 
    \end{pmatrix}$

    Если $A$ состоит из нулей, то мы уже имеем ступенчатый вид, поэтому мы будем считать что не все её элементы нулевые. Пусть $C_i(A) := \begin{pmatrix}
        a_{1j} \\ a_{2j} \\ \vdots \\ a_{mj}
    \end{pmatrix}$ - самый левый ненулевой столбец. Если $a_{1j} = 0$, то с помощью перестановки строк добиваемся того, чтобы первый столбец был ненулевой. Теперь с помощью $a_{1j}$ зануляем все элементы, ниже его в этом столбце ($R_m(A) \rightarrow R_m(A) - \frac{a_{mj}}{a_{1j}}R_1(A)$ и т.д.) теперь наша матрица примет вид:

    $\begin{pmatrix}
    0 & \dots & 0 & a_{1j}     & a_{1, j+1} & \dots & a_{1n} \\ 
    0 & \dots & 0 & 0     & a'_{2, j+1} & \dots & a_{1n} \\ 
    \vdots & \ddots & \vdots & \vdots     & \vdots & \ddots & \vdots \\ 
    0 & \dots & 0 & 0     & a'_{m, j+1} & \dots & a_{mn} \\ 
    \end{pmatrix}$

    Проделаем такую же процедуру с матрицей $A := \begin{pmatrix}
        a'_{2, j+1} & \dots & a_{1n} \\ 
        \vdots     &  \ddots & \vdots \\
        a'_{m, j+1} & \dots & a_{mn} \\ 
    \end{pmatrix}$

    Будем продолжать, пока не получим ступенчатый вид. Обозначим получивш. матрицу за $B$, пусть ее элементы - $b_{ij}$

    $b_{1, j1}, b_{2, j_2}, ... , b_{j, jk}$ - ведущие элементы. Делим первую строку на $b_{1, j1}$, вторую на $b_{2, j2}$ и т.д. Получаем, что в полученной $B'$ все ведующие элементы равны 1. С помощью этих элементов и сложением строк зануляем элементы над ведущими. Получаем редуцированный ступенчатый вид. Гойда гол.
    
\end{quote}

УРА ПРОСТРАНСТВА \\

(3) В любом векторном пространстве $\mathbf{V}$ верны следующие утверждения:

(a) вектор $\mathbf{o}$ – единствен 
\begin{quote}
Пусть $o'$ и $o$ - нулевые:
    $o' = o + o' = o$
\end{quote}

(b) для любого вектора $\mathbf{a} \in \mathbf{V}$ имеется только один обратный к нему
\begin{quote}
   Пусть для $a$ есть два обратных: $b$ и $c$: 

   $
   c = c + o = c + (a + b) = (c + a) + b = o + b = b
   $
\end{quote}
(c) $\alpha \cdot \mathbf{o} = \mathbf{o}$
\begin{quote}
    $v = v + o$, с др. стороны $o = \alpha v + (-(\alpha v))$

    $v = v + 0 \Rightarrow \alpha v = \alpha v + \alpha o$

    $o = \alpha v + (- \alpha v) = \alpha v + \alpha o + (-\alpha v) = \alpha v + (- \alpha v) + \alpha o = o + \alpha o = \alpha o$
    $o = \alpha v + (- \alpha v) = \alpha v + \alpha o + (-\alpha v) = \alpha v + (- \alpha v) + \alpha o = o + \alpha o = \alpha o$
\end{quote}

(d) $\alpha \cdot (-\mathbf{v}) = - \alpha \mathbf{v}$
\begin{quote}
    $\alpha v + \alpha \cdot (-v) + (- \alpha v) = (\alpha v + \alpha (-v)) + (- \alpha v) =
    (\alpha(v + (-v)) + (-\alpha v) = \alpha o + (- \alpha v) = o + (-\alpha v) = -\alpha v$
\end{quote}
Похуй гойда.

(4) Векторное подпространство является векторным пространством относительно тех же операций которые определены в изначальном пространстве. Докажите. 
\begin{quote}
    Пусть $V$ - векторное пространство, а $W$ - его подпространство. $W \subseteq V \Longleftrightarrow$ все элементы $W$ - векторы из $V$. Сумма векторов из $W$ лежит в $W$, значит она лежит и в $V$. Значит по аксиомам векторного пространства выполняются ассоциативность и коммутативность сложения (1 и 2). Т.к. $o \in W$ по опр., выполняется аксиома существования нуля (3). Произведение вектора на скаляр лежит в $W$, значит лежит и в $V$. Тогда выполняются все оставшиеся аксиомы. GOIDA SVO!
\end{quote}

(5) Множество всех решений какой-то однородной линейной системы от $n$ перемен- ных есть подпространство в $\mathbb{R}^n$. Докажите. 
\begin{quote}
    Дана система со множеством решений $S$: $\begin{cases}
        a_{11}x_1 + ... + a_{1n}x_n = 0 \\
        ... \\
        a_{m1}x_1 + ... + a_{mn}x_n = 0 \\
    \end{cases}$
    
    \begin{enumerate}
        \item $o = (0, ..., 0)^T$ является решением системы, т.е. $o \in S$
        \item Пусть $s = (s_1, ..., s_n)^T, s' = (s'_1, ..., s'_n)^T$ - решения системы:\\
        $
        \begin{cases}
        a_{11}s_1 + ... + a_{1n}s_n \equiv 0 \\
        ... \\
        a_{m1}s_1 + ... + a_{mn}s_n \equiv 0 \\
        \end{cases}
        $
        $
        \begin{cases}
        a_{11}s'_1 + ... + a_{1n}s'_n \equiv 0 \\
        ... \\
        a_{m1}s'_1 + ... + a_{mn}s'_n \equiv 0 \\
        \end{cases}
        $
        
        $\Rightarrow$
        $
        \begin{cases}
        a_{11}(s_1 + s'_1) + ... + a_{1n}(s_n + s'_n) \equiv 0 \\
        ... \\
        a_{m1}(s_1 + s'_1) + ... + a_{mn}(s_n + s'_n) \equiv 0 \\
        \end{cases}
        $

        Т.е. $s + s'$ - решение системы.

        \item 
        $
        \begin{cases}
        a_{11}(\lambda s_1) + ... + a_{1n}(\lambda s_n) \equiv 0 \\
        ... \\
        a_{m1}(\lambda s_1) + ... + a_{mn}(\lambda s_n) \equiv 0 \\
        \end{cases}
        \Rightarrow 
        \begin{cases}
        \lambda(a_{11}s_1 + ... + a_{1n}s_n) \equiv 0 \cdot \lambda \\
        ... \\
        \lambda(a_{m1}s_1 + ... + a_{mn}s_n) \equiv 0 \cdot \lambda \\
        \end{cases}
        $
        Т.е. $\lambda s$ - решение системы.
    \end{enumerate}
\end{quote}

(6) Линейная оболочка является векторным пространством. Докажите. 
\begin{quote}
    Пусть $S = \{s_1, ..., s_m \} \not= \varnothing$, $S \subseteq V$

    Докажем, что $Span_{\mathbb{R}}(S)$ - подпространство в $V$. Тогда из (4) будет вытекать утверждение.

    Рассмотрим $v, w \in Span_{\mathbb{R}}(S)$, тогда $v = \alpha_1 s_1 + ... + \alpha_m s_m, w = \beta_1 s_1 + ... + \beta s_m$, согласно опр. вект. простр., $v, w \in V$. Тогда по аксиомам векторного пространства:

    $v + w = (\alpha_1 s_1 + ... + \alpha_m s_m) + (\beta_1 s_1 + ... + \beta_m s_m) = (\alpha_1 s_1 + \beta_1 s_1) + ... + (\alpha_m s_m + \beta_m s_m) = (\alpha_1 + \beta_1) s_1 + (\alpha_m + \beta_m) s_m$, т.е. $v + w \in Span_{\mathbb{R}}(S)$

    $\alpha \cdot v = \alpha \cdot (\alpha_1 s_1 + ... + \alpha_m s_m) = \alpha (\alpha_1 s_1) + ... + \alpha (\alpha_m s_m) = (\alpha \alpha_1) s_1 + ... + (\alpha \alpha_m) s_m$ т.е. $\alpha v \in Span_{\mathbb{R}}(S)$

    Мы можем взять $\alpha_1 = \alpha_2 = ... = \alpha_m = 0$, тогда получим нулевой вектор. Т.е. $o \in Span_{\mathbb{R}}(S)$
\end{quote}

(7) У данного вектора в данном базисе имеется только единственные координаты, то есть у одного и того же вектора в данном базисе не может быть две разные координаты. Докажите.
\begin{quote}
   Пусть это не так, тогда у $w$ есть две координаты: $(\alpha_1 , ... \alpha_n)^T$ и $(\alpha'_1, ..., \alpha'_n)^T$ 

   $w = \alpha_1 e_1 + ... + \alpha_n e_n$,
   $w = \alpha'_1 e_1 + ... + \alpha'_n e_n$

   Тогда $w - w = \alpha_1 e_1 + ... + \alpha_n e_n - \alpha'_1 e_1 - ... - \alpha'_n e_n = (\alpha_1 - \alpha'_1)e_1 + ... + (\alpha_n - \alpha'_n)e_n$

   $w - w = o = (0, ..., 0)^T$, т.е. $\alpha_1 - \alpha'_1 = 0, ..., \alpha_n - \alpha'_n = 0$, т.е. $\alpha_1 = \alpha'_1 ...$ ГОЛ
\end{quote}

(8) В пространстве $\mathbb{R}^n$ рассмотрим следующие $n$ векторов $e_1 = (1,0, … , 0), \ e_2 = (0,1, … , 0), ⋯ ⋯ ⋯, \ e_n = (0,0, … , 1)$ Эти векторы образуют базис пространства $\mathbb{R}^n$ который принято называть стан- дартным базисом. Докажите. 
\begin{quote}
    Докажем, что $\mathbb{R}^n$ - ЛО, натянутая на эти вектора:

    $\begin{pmatrix}
        \alpha_1 \\ \alpha_2 \\ \vdots \\ \alpha_n
    \end{pmatrix} = \alpha_1 \cdot e_1 + \alpha_2 \cdot e_2 + ... + \alpha_n \cdot e_n$

    Докажем, что эти вектора ЛНЗ:
    $
    x_1 \cdot \begin{pmatrix}
        1 \\ 0 \\ \vdots \\ 0
    \end{pmatrix} +     x_2 \cdot \begin{pmatrix}
        0 \\ 1 \\ \vdots \\ 0
    \end{pmatrix} + ... +    x_n \cdot \begin{pmatrix}
        0 \\ 0 \\ \vdots \\ 1
    \end{pmatrix} = \begin{pmatrix}
        0 \\ 0 \\ \vdots \\ 0
    \end{pmatrix}
    $

    Чтобы первый элемент был нулевым, нужно, чтобы $x_1$ был нулевым, и т.д. до $x_n$, т.е. решением будет является только тривиальное, т.е. вектора ЛНЗ.
\end{quote}

(9) Если $\mathbf{V}$ – векторное пространство размерности $n$, то оно изоморфно пространству $\mathbf{R}^n$. Докажите.
\begin{quote}
    Пусть $E = \{ e_1 , ... , e_n \}$ - базис в $V$,
    тогда пусть:

    $\varphi(e_1) := (1, 0, 0, ..., 0)^T$ 
    
    $\varphi(e_2) := (0, 1, 0, ..., 0)^T$ 

    ...
    
    $\varphi(e_n) := (0, 0, 0, ..., 1)^T$ 

    Т.к. любой вектор $v \in V$ имеет какие-то координаты в $E$, скажем $v = (\alpha_1, ... , \alpha_n)^T$, т.е:
    
    $v = \alpha_1 e_1 + ... + \alpha_n e_n$, пусть $\varphi$ - линейно, т.е. $\varphi (v) := \varphi (\alpha_1 e_1 + ... + \alpha_n e_n) = \alpha_1 \varphi (e_1) + ... + \alpha_n \varphi (e_n) = \alpha_1 \begin{pmatrix}
        1 \\ \vdots \\ 0
    \end{pmatrix} + ... + \alpha_n \begin{pmatrix}
        0 \\ \vdots \\ 1
    \end{pmatrix} = \begin{pmatrix}
        \alpha_1 \\ \vdots \\ a_n
    \end{pmatrix} \in \mathbb{R}^n$

    Таким образом, для любого вектора $(\alpha_1, ..., \alpha_n)^T \in \mathbb{R}^n$ можно найти вектор $v \in V$, у которого в базисе $E$ координаты $\alpha_1, ... , \alpha_n$, такой что $\varphi(v) = (\alpha_1, ..., \alpha_n)^T$. Т.е. $\varphi$ - сюръективно.

    Возьмем $v, v' \in V$ с коорд. $\alpha_1, ..., \alpha_n$ и $\alpha'_1, ... , \alpha'_n$ соотв. в базисе $E$:

    $\varphi(v) = \begin{pmatrix}
        \alpha_1 \\ \vdots \\ \alpha_n
    \end{pmatrix}$, \
        $\varphi(v') = \begin{pmatrix}
        \alpha'_1 \\ \vdots \\ \alpha_n
    \end{pmatrix}$

    Если $\varphi (v) = \varphi (v')$, то:  $\begin{pmatrix}
        \alpha_1 \\ \vdots \\ \alpha_n
    \end{pmatrix} = 
        \begin{pmatrix}
        \alpha'_1 \\ \vdots \\ \alpha_n
    \end{pmatrix}$, то есть $a_1 = a'_1, ..., a_n = a'_n$, т.е. $v = v'$

    Значит, $\varphi$ - инъективно. Т.е. мы построили изоморфизм. Гойда гол!
\end{quote}
\end{document}