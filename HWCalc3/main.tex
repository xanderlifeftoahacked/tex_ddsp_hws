\documentclass{article}
\usepackage{ucs} 
\usepackage[utf8x]{inputenc} % Включаем поддержку UTF8  
\usepackage{amsmath,amssymb}
\usepackage{tikz}
\usepackage{setspace}
\usepackage{amsfonts}
\usepackage{geometry}
\usepackage{quoting}
\usepackage[russian]{babel}  % Включаем пакет для поддержки русского языка  
\usepackage[a4paper, left=0.5cm, right=0.5cm, top=0cm, bottom=0cm]{geometry}
\begin{document}
\setlength{\parindent}{0pt}
\begin{Large}
    \textsf{\textbf{МатАн ДЗ №3}}
    
    Шамаев Александр    
\end{Large}
\vspace{1cm}


\textsf{\textbf{(1)}}

\begin{quote}
(a) $\{x_n\} = \{0 , 1 , 1 , 1 ...\}$, \  $\{y_n\} = \{1 , 0 , 0 , 0 ...\}$

(b) $\{x_n\} = \{0 , 1 , 1 , 1 ...\}$, \  $\{y_n\} = \{1 , 0 , 1 , 0 , 1 ...\}$
\end{quote}

\textsf{\textbf{(2)}}

\begin{quote}
(a) $a_n = a$

(b) $a_n  = a + \frac{1}{n}$


(c) $a_n = \begin{cases}
a , \quad n\text{ - чет.} \\
\frac{1}{n} + a , \quad n\text{ - нечет.}
\end{cases}$

(d) $\{a_n\} = \{a, a - 1, a, a - 1...\}$
\end{quote}

\textsf{\textbf{(3)}}

\begin{quote}
Если $x$ - предел \{$x_n$\}, значит (по определению) в любой окрестности $x$ содержатся почти все члены $\{x_n\}$, т.е. все элементы за исключением
их конечного числа. Если мы добавим в начало этой последовательности конечное число элементов, то в любой окрестности $x$ по-прежнему будут содержатся почти все элементы, значит $x$ по-прежнему будет пределом \{$x_n$\}. Аналогично с отбрасыванием конечного числа элементов.  
\end{quote}

\textsf{\textbf{(4)}}
\begin{quote}
(a) 
\begin{quote}
   Из определения предела:
   $\forall \varepsilon > 0 \ \exists N \ \forall n > N \ : \ |\frac{1}{\sqrt{3n - 11}} - 0| < \varepsilon$
   
    $\sqrt{3n - 11} > 0$ при натуральных $n$ $\Leftrightarrow $ дробь всегда положительна, тогда:
    
    $\frac{1}{\sqrt{3n - 11}} < \varepsilon \Longleftrightarrow
    3n - 11 > \frac{1}{\varepsilon ^ 2} \Longleftrightarrow n > \frac{1}{3 \varepsilon ^ 2} + \frac{11}{3}$

    Неравенство будет выполнятся для $\forall n > N$ при $N = \left\lceil{\frac{1}{3 \varepsilon ^ 2} + \frac{11}{3}}\right\rceil + 1$. Сходимость доказана.
\end{quote}

(b) 
\begin{quote}
   Из определения предела:
   $\forall \varepsilon > 0 \ \exists N \ \forall n > N \ : \ |\frac{2n + 3}{n ^ 2} - 0| < \varepsilon$
   
    $n^2 > 0, \ 2n + 3 > 0$ при натуральных $n$, т.е. дробь всегда положительна, тогда:

    $\frac{2n + 3}{n^2} = \frac{2}{n} + \frac{3}{n ^ 2} < \frac{2}{n} + \frac{3}{n} < \varepsilon \Longleftrightarrow n > \frac{5}{\varepsilon}$

    
    Неравенство будет выполнятся для $\forall n > N$ при $N = \left\lceil{\frac{5}{\varepsilon}}\right\rceil$. Сходимость доказана.
\end{quote}

(c) 
\begin{quote}
   Из определения предела:
   $\forall \varepsilon > 0 \ \exists N \ \forall n > N \ : \ |\frac{\cos n}{\sqrt{n}} - 0| < \varepsilon$

   
   Косинус принимает значения от -1 до 1, значит дробь можем ограничить сверху так:
   $
    |\frac{\cos n}{\sqrt{n}}| \leq \frac{1}{\sqrt{n}}
   $

  $\frac{1}{\sqrt{n}} < \varepsilon \Longleftrightarrow n > \frac{1}{\varepsilon ^ 2}$ 
  
    Неравенство будет выполнятся для $\forall n > N$ при $N = \left\lceil{\frac{1}{\varepsilon ^ 2}}\right\rceil$. Сходимость доказана.
\end{quote}
\end{quote}

\textbf{\textsf{(5)}}
\begin{quote}
$\lim\limits_{n\to \infty} x_n = - \infty$ \qquad
$\exists C \ \exists n_0(C) \ \forall n > n_0(C) : \ y_n \leq C$

(?) $\lim\limits_{n\to \infty}(x_n + y_n) = -\infty$

$x_n$ - стремится к $- \infty$, по определению:

$\forall M > 0 \ \exists N(M) \ \forall n > N(M): x_n < -M$


$x_n + y_n < -M + y_n$ \ , если $N > n_0$, то $x_n + y_n < -M + C$

Возьмем $N_1 = max(N(M), n_0(C))$. 
Тогда: 

$\forall \underbrace{(-M + C)}_{M_1} > 0 \ \exists N_1 \ \forall n > N_1: x_n + y_n < \underbrace{-M + C}_{M_1}$

Т.е. $x_n + y_n$ - стремится к $- \infty$ по опр. ЧТД.


\end{quote}

\textbf{\textsf{(6)}}

\begin{quote}

$\lim\limits_{n\to \infty} x_n = \infty$ \qquad
(*) $\exists C \ \exists n_0(C) \ \forall n > n_0(C) : \ y_n \geq C > 0$

(?) $\lim\limits_{n \to \infty} (x_n \cdot y_n) = a$

1. $a = + \infty$
\begin{quote}
    
$x_n$ стремится к $+\infty$, по определению:

\begin{cases}
    
$\forall M > 0 \ \exists N(M) \ \forall n > N(M): x_n > M$ \\
(*)
\end{cases}
$\Rightarrow$

$\Rightarrow$
$\forall M > 0 \ \exists N \ \forall n > \underbrace{(N + n_0)}_{N'}: x_n \cdot y_n > \underbrace{M \cdot C}_{M'}$

$\forall M' > 0 \ \exists N' \ \forall n > N': x_n \cdot y_n > M' \Leftrightarrow x_n \cdot y_n$ стремится к $+ \infty$ по опр.
\end{quote}
2. $a = - \infty$
\begin{quote}
    
$x_n$ стремится к $-\infty$, по определению:

\begin{cases}
    
$\forall M > 0 \ \exists N(M) \ \forall n > N(M): x_n < -M$ \\
(*)
\end{cases}
$\Rightarrow$

$\Rightarrow$
$\forall M > 0 \ \exists N \ \forall n > \underbrace{(N + n_0)}_{N'}: x_n \cdot y_n < \underbrace{-M \cdot C}_{M'}$

$\forall M' > 0 \ \exists N' \ \forall n > N': x_n \cdot y_n < -M' \Leftrightarrow x_n \cdot y_n$ стремится к $- \infty$ по опр.

\end{quote}
ЧТД.

\end{document}
