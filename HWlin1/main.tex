\documentclass{article}
\usepackage{ucs} 
\usepackage[utf8x]{inputenc} % Включаем поддержку UTF8  
\usepackage{amsmath}
\usepackage{tikz}
\usepackage{setspace}
\usepackage{amsfonts}
\usepackage{geometry}
\usepackage{quoting}
\usepackage[russian]{babel}  % Включаем пакет для поддержки русского языка  
\usepackage[a4paper, left=0.5cm, right=0.5cm, top=0cm, bottom=0cm]{geometry}


\begin{document}
\setlength{\parindent}{0pt}
\begin{Large}
    \textsf{\textbf{ЛинАл ДЗ №1}}
    
    Шамаев Александр    
\end{Large}
\vspace{1cm}
\\ \textsf{\textbf{(1)}}

\hspace{1cm}(a)
$
\begin{pmatrix}
1 & 5 & | & 7 \\
-2 & -7 & | & -5 \\
\end{pmatrix} \Longleftrightarrow
\begin{pmatrix}
    1 & 5 & | & 7 \\
    0 & 3 & | & 9 \\
\end{pmatrix} \Longleftrightarrow
\begin{pmatrix}
    1 & 5 & | & 7 \\
    0 & 1 & | & 3 \\
\end{pmatrix} \Longleftrightarrow
\begin{pmatrix}
    1 & 0 & | & -8 \\
    0 & 1 & | & 3 \\
\end{pmatrix}
$

\begin{quote}
\leftskip=0.15cm    
\textbf{Ответ: $(-8, 3)$\\}

\end{quote}
\hspace{1cm}(b)
$
\begin{pmatrix}
    0 & 1 & 4 & | & -5 \\
    1 & 3 & 5 & | & -2 \\
    3 & 7 & 7 & | & 6 \\
\end{pmatrix} \Longleftrightarrow
\begin{pmatrix}
    1 & 3 & 5 & | & -2 \\
    3 & 7 & 7 & | & 6 \\
    0 & 1 & 4 & | & -5 \\
\end{pmatrix} \Longleftrightarrow
\begin{pmatrix}
    1 & 3 & 5 & | & -2 \\
    0 & -2 & -8 & | & 12 \\
    0 & 1 & 4 & | & -5 \\
\end{pmatrix} \Longleftrightarrow
\begin{pmatrix}
    1 & 3 & 5 & | & -2 \\
    0 & 1 & 4 & | & -6 \\
    0 & 1 & 4 & | & -5 \\
\end{pmatrix} 

$

\begin{quote}
\leftskip=0.15cm
Из второй и третьей строчки: $-5 = -6$.     

\textbf{Ответ: $\emptyset$\\}

\end{quote}%

\hspace{1cm} (c)
$
\begin{pmatrix}
    1 & -3 & 4 & | & -4 \\
    3 & -7 & 7 & | & -8 \\
    4 & 6 & -1 & | & 7 \\
\end{pmatrix} \Longleftrightarrow
\begin{pmatrix}
    1 & -3 & 4 & | & -4 \\
    0 & 2 & -5 & | & 4 \\
    4 & 6 & -1 & | & 7 \\
\end{pmatrix} \Longleftrightarrow
\begin{pmatrix}
    1 & -3 & 4 & | & -4 \\
    0 & 2 & -5 & | & 4 \\
    0 & 18 & -17 & | & 23 \\
\end{pmatrix} \Longleftrightarrow

\hspace{1cm} \Longleftrightarrow
\begin{pmatrix}
    1 & -3 & 4 & | & -4 \\
    0 & 2 & -5 & | & 4 \\
    0 & 0 & 28 & | & -13 \\
\end{pmatrix} \Longleftrightarrow
\begin{pmatrix}
    1 & -3 & 4 & | & -4 \\
    0 & 1 & -\frac{5}{2} & | & 2 \\
    0 & 0 & 1 & | & -\frac{13}{28} \\
\end{pmatrix}
$
    
\begin{quote}
 
\leftskip=0.15cm
Подставим $x_3$ во второе уравнение: $x_2 - \frac{5}{2} * -\frac{13}{28} = 2$ \Longleftrightarrow $x_2 = 2 - \frac{65}{56} = \frac{47}{56}$.
Подставим $x_2$ и $x_3$ в первое уравнение:
$x_1 - 3 * \frac{47}{56} + 4 * -\frac{13}{28} = -4 $ \Longleftrightarrow
$x_1 = -4 + \frac{13}{7} + \frac{141}{56} = \frac{3}{8}$

\textbf{Ответ: $(\frac{3}{8}, \frac{47}{56}, -\frac{13}{28})$\\}

\end{quote}    
\hspace{1cm}(d)
$\begin{pmatrix}
    1 & 0 & 3 & 0 & | & 2 \\
    0 & 1 & 0 & -3 & | & 3 \\
    0 & -2 & 3 & 2 & | & 1 \\
    3 & 0 & 0 & -12 & | & -5 \\
\end{pmatrix} \Longleftrightarrow
\begin{pmatrix}
    1 & 0 & 3 & 0 & | & 2 \\
    0 & 1 & 0 & -3 & | & 3 \\
    0 & -2 & 3 & 2 & | & 1 \\
    0 & 0 & -9 & -12 & | & -11 \\
\end{pmatrix}$ \Longleftrightarrow
\begin{pmatrix}
    1 & 0 & 3 & 0 & | & 2 \\
    0 & 1 & 0 & -3 & | & 3 \\
    0 & 0 & 3 & -4 & | & 7 \\
    0 & 0 & -9 & -12 & | & -11 \\
\end{pmatrix} \Longleftrightarrow

\hspace{1cm}
\Longleftrightarrow \begin{pmatrix}
    1 & 0 & 3 & 0 & | & 2 \\
    0 & 1 & 0 & -3 & | & 3 \\
    0 & 0 & 3 & -4 & | & 7 \\
    0 & 0 & 0 & -24 & | & 10 \\
\end{pmatrix}
\Longleftrightarrow \begin{pmatrix}
    1 & 0 & 3 & 0 & | & 2 \\
    0 & 1 & 0 & -3 & | & 3 \\
    0 & 0 & 1 & -\frac{4}{3} & | & \frac{7}{3} \\
    0 & 0 & 0 & 1 & | & -\frac{5}{12} \\
\end{pmatrix}
$

\begin{quote}
   \leftskip=0.15cm 
    
Подставим $x_4$ в третье уравнение: $x_3 = \frac{4}{3} * -\frac{5}{12} + \frac{7}{3} = \frac{16}{9}$

Подставим $x_4$ во второе уравнение: $x_2 = 3 * -\frac{5}{12} + 3 = \frac{7}{4} $

Подставим $x_3$ в первое уравнение: $x_1 =  - 3 * \frac{16}{9}  + 2= -\frac{10}{3}$


\textbf{Ответ: $(-\frac{10}{3}, \frac{7}{4}, \frac{16}{9}, -\frac{5}{12})$\\}
\end{quote}

\textsf{\textbf{(2)}}

\hspace{1cm}(a) 
$
\begin{pmatrix}
   1 & h & | & 4 \\
   3 & 6 & | & 8 \\
\end{pmatrix} \Longleftrightarrow
\begin{pmatrix}
   1 & h & | & 4 \\
   0 & 6 - 3h & | & -4 \\
\end{pmatrix}
$

\begin{quote}
\leftskip=0.15cm
Система имеет решения, когда  $ 6 - 3h \neq 0 $ 

(в ином случае, второе ур. превращается в $0 = -4$, что неверно)

\textbf{Ответ: $h \neq 2$\\}

\end{quote}

\hspace{1cm}(b)
\begin{pmatrix}
    2 & -3 & | & h \\
    -6 & 9 & | & 5 \\
\end{pmatrix} \Longleftrightarrow
\begin{pmatrix}
    2 & -3 & | & h \\
    0 & 0 & | & 5 + 3h \\
\end{pmatrix}


\begin{quote}
\leftskip=0.15cm    
    Система имеет решения, когда $5 + 3h = 0$ 
    
    \textbf{Ответ: $h = -\frac{5}{3}$\\}


\end{quote}

\hspace{1cm}(c)
\begin{pmatrix}
    1 & h & | & -3 \\
    2 & 4 & | & 6 \\
\end{pmatrix} \Longleftrightarrow
\begin{pmatrix}
    1 & h & | & -3 \\
    0 & 4 - 2h & | & 12 \\
\end{pmatrix}

\begin{quote}
\leftskip=0.15cm    
    Система имеет решения, когда $4 - 2h \neq 0$ 
    
    \textbf{Ответ: $h \neq 2$\\}

\end{quote}


\textsf{\textbf{(3)}}

\hspace{1cm}
    \begin{pmatrix}
        1 & 3 & | & f \\
        c & d & | & g \\ 
    \end{pmatrix} \Longleftrightarrow
     \begin{pmatrix}
        1 & 3 & | & f \\
        0 & d - 3c & | & g - fc \\ 
    \end{pmatrix}

\begin{quote}
    
Чтобы система имела решения при любых $f,g \in \mathbb{R}$,
надо, чтобы второе уравнение 
не превратилось в ложное выражение (например, 0 = 1). Его правая часть м.б. любой, значит 
левая должна быть ненулевой, т.е. $d \neq 3c$


    \textbf{Ответ: $d \neq 3c$\\}

\end{quote}
\textsf{\textbf{(4)}}

\begin{quote}
(a) Поменять местами $r_2$ и $r_1$

(b) $r_2 = r_2 * -\frac{1}{2}$

(c) $r_3 = r_3 - r_1 * 4$ 

(d) $r_3 = r_3 + r_2 * 3$

\end{quote}

\empty


\textsf{\textbf{(5)}}


\hspace{1cm}Составим систему уравнений:
\begin{equation*}
 \begin{cases}
 T_1 = \frac{10 + 20 + T_2 + T_4}{4}
   \\
   
 T_2 = \frac{40 + 20 + T_3 + T_1}{4}
   \\
   
 T_3 = \frac{40 + 30 + T_4 + T_2}{4}
 \\
 T_4 = \frac{10 + 30 + T_1 + T_3}{4}
 \end{cases} \Longleftrightarrow
 \begin{cases}
  4T_1 - T_2 - T_4 = 30
   \\
  - T_1 + 4T_2 - T_3= 60
   \\
  - T_2 +  4T_3 - T_4= 70
 \\
 - T_1 - T_3 + 4T_4= 40
 \end{cases}
\end{equation*}

\hspace{1cm}Решим ее: 
$\begin{pmatrix}
    4 & -1 & 0 & -1 & | & 30 \\
    -1 & 4 & -1 & 0 & | & 60 \\
    0 & -1 & 4 & -1 & | & 70 \\
    -1 & 0 & -1 & 4 & | & 40 \\
\end{pmatrix}$ \Longleftrightarrow
$\begin{pmatrix}
    4 & -1 & 0 & -1 & | & 30 \\
    0 & \frac{15}{4} & -1 & -\frac{1}{4} & | &\frac{135}{2}  \\
    0 & -1 & 4 & -1 & | & 70 \\
    -1 & 0 & -1 & 4 & | & 40 \\
\end{pmatrix}$ \Longleftrightarrow

\hspace{2cm}\Longleftrightarrow
$\begin{pmatrix}
    4 & -1 & 0 & -1 & | & 30 \\
    0 & \frac{15}{4} & -1 & -\frac{1}{4} & | &\frac{135}{2}  \\
    0 & -1 & 4 & -1 & | & 70 \\
    0 & -\frac{1}{4} & -1 & \frac{15}{4} & | & \frac{95}{2} \\
\end{pmatrix}$ \Longleftrightarrow
$\begin{pmatrix}
    4 & -1 & 0 & -1 & | & 30 \\
    0 & \frac{15}{4} & -1 & -\frac{1}{4} & | &\frac{135}{2}  \\
    0 & -1 & 4 & -1 & | & 70 \\
    0 & 0 & -2 & 4 & | & 30 \\
\end{pmatrix}$ \Longleftrightarrow

\hspace{2cm}\Longleftrightarrow
$\begin{pmatrix}
    4 & -1 & 0 & -1 & | & 30 \\
    0 & \frac{15}{4} & -1 & -\frac{1}{4} & | &\frac{135}{2}  \\
    0 & 0 & \frac{56}{15} & -\frac{16}{15} & | & 88 \\
    0 & 0 & -2 & 4 & | & 30 \\
\end{pmatrix}$ \Longleftrightarrow
$\begin{pmatrix}
    4 & -1 & 0 & -1 & | & 30 \\
    0 & \frac{15}{4} & -1 & -\frac{1}{4} & | &\frac{135}{2}  \\
    0 & 0 & \frac{56}{15} & -\frac{16}{15} & | & 88 \\
    0 & 0 & 0 & \frac{24}{7} & | & \frac{540}{7}\\
\end{pmatrix}$

\hspace{1cm}Получается, $x_4 = \frac{540}{24} = \frac{45}{2}.$ 
Подставим в третье уравнение: $\frac{56}{15}x_3 = 88 + \frac{16}{15} * \frac{45}{2}= 

\hspace{1cm}= 112\Longleftrightarrow x_3 = 30.$ Подставим во второе: 
$15x_2 = 4 * 30 + \frac{45}{2} + 270 \Longleftrightarrow x_2 = \frac{55}{2}.$

\hspace{1cm}Подставим в первое: $4x_1 = \frac{55}{2} + \frac{45}{2} + 30 = 80 \Longleftrightarrow x_1 = 20$

\begin{quote}
\leftskip=0.1cm
    \textbf{Ответ: $(20, \frac{55}{2}, 30, \frac{45}{2})$}

\end{quote}
\end{document}