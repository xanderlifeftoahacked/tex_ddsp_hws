
\documentclass{article}
\usepackage{ucs} 
\usepackage[utf8x]{inputenc} % Включаем поддержку UTF8  
\usepackage{amsmath}
\usepackage{tikz}
\usepackage{amssymb}
\usepackage{setspace}
\usepackage{amsfonts}
\usepackage{geometry}
\usepackage{quoting}
\usepackage[russian]{babel}  % Включаем пакет для поддержки русского языка  
\usepackage[a4paper, left=0.5cm, right=0.5cm, top=0cm, bottom=0cm]{geometry}
\begin{document}
\setlength{\parindent}{0pt}
\begin{Large}
    \textsf{\textbf{ДемоКР 1 МатАн: найди ошибку}}
    
\end{Large}
\vspace{1cm}

\textbf{\textsf{(1)}}
\begin{quote}
    a) 
    \begin{quote}
        
    База: 0 = 0  - верно
    
    Пусть верно для $n = k$: $1 \cdot 2 + 2 \cdot 3 + ... + (k - 1) \cdot k = \frac{(k - 1)k(k + 1)}{3}$
    
    Докажем шаг $k \rightarrow k + 1$: $\underbrace{1 \cdot 2 + 2 \cdot 3 + ... + (k - 1)k}_{= \frac{(k - 1)k(k + 1)}{3}} + k(k + 1) = \frac{(k - 1)k(k + 1) + 3k(k + 1)}{3} = $
    $= \frac{k(k + 1)(k - 1 + 3)}{3} =  \frac{((k + 1) - 1)(k + 1)((k + 1) + 1)}{3}$ $\quad$ ЧТД
    \end{quote}

    b) 
    \begin{quote}
    
        База: $7 + 12 + 17 \equiv 0 \mod 18$ - верно
        
        Пусть верно для $n = k$: $7^k + 12k + 17 \equiv 0 \mod 18$
        
        Докажем шаг $k \rightarrow k + 1$: $7 \cdot 7^k + 12 + 12k + 17 = \underbrace{7^k + 12k + 17}_{\equiv 0 \mod 18} + 6 \cdot 7^k + 12$

        Осталось доказать, что $6 \cdot 7^k + 12 \equiv 0 \mod 18$
        
        База: $42 + 12 \equiv 0 \mod 18$

        Пусть верно для $n = k$: $6 \cdot 7^k + 12 \equiv 0 \mod 18$

        Докажем шаг $k \rightarrow k + 1$: $7 \cdot 6 \cdot 7^k + 12 = \underbrace{36 \cdot 7^k}_{\equiv 0 \mod 18} + \underbrace{6 \cdot 7^k + 12}_{\equiv 0 \mod 18}$

        ЧТД
    \end{quote}
    
    
    \end{quote} 

    
\textbf{\textsf{(2)}}

\begin{quote}
    
Второе слагаемое = $C^1_n \cdot (\sqrt[5]{x^2})^{n - 1} \cdot \frac{-1}{2 \cdot \sqrt[6]{x}}$ $\quad$ коэф. = $C^1_n \cdot \frac{-1}{2} = \frac{-n}{2}$

Третье слагаемое = $C^2_n \cdot (\sqrt[5]{x^2})^{n - 2} \cdot \frac{-1}{2 \cdot \sqrt[6]{x}}^2$ $\quad$ коэф. = $C^2_n \cdot \frac{1}{2^2} = \frac{n(n - 1)}{8}$

$\frac{n(n - 1)}{8} - \frac{n}{2} = \frac{51}{2} \Leftrightarrow  n^2 - n - 4n = 204 \Leftrightarrow (n + 12)(n - 17) = 0$

$n \in \mathbb{N} \Rightarrow n = 17$ 

Найдем при каком $k$, член не будет содержать $x$ (пока что опустим коэф.): 

$(x^{\frac{2}{5}})^{17 - k} \cdot (x^{-\frac{1}{6}})^k = 1 \Leftrightarrow$
$\frac{2}{5} \cdot (17 - k) = -(-\frac{1}{6}) \cdot k \Leftrightarrow$
$204 - 12k = 5k \Leftrightarrow$
$k = 12$

Теперь найдем коэфицент при этом члене: $C^{12}_{17} \cdot (-\frac{1}{2})^{12} = C^{5}_{17} \cdot \frac{1}{2^{12}} = \frac{17 \cdot 16 \cdot 15 \cdot 14 \cdot 13}{2 \cdot 3 \cdot 4 \cdot 5 \cdot 2^{12}} = \ = \boxed{\frac{1547}{1024}}$
\end{quote}

\textbf{\textsf{(3)}}
\begin{quote}
(a)
\begin{quote}
    
$\forall C > 0\ \exists n: \ |a_n| > C  $

$|2n^2 - n - 1| \geq 2n^2 - n - 1 > n^2 - 4n - 4 > C$

$(n - 2)^2 > C$

$|n - 2| > \sqrt{C}$

$n > \sqrt{C} + 2$

Неограниченность доказана.
\end{quote}
(b)
\begin{quote}
$\forall C > 0\ \exists n: \ |b_n| > C  $

$|\frac{n! + n - 3^n}{n^2 + 7n}| \geq \frac{n! + n - 3^n}{n^2 + 7n} > \frac{n! - 3^n} {8n^2} > C$

$n! - 3^n - 8Cn^2 > n! - 3^n - 8C\cdot3^n = n! - 3^n(8C + 1) \overset{n > 9}{>} n! - (n - 1)!(8C + 1) = \ = (n-1)!(n - 8C - 1) > 0$

\begin{cases}
    $
   n > 9 \\ 
   n > 8C + 1
    $
\end{cases}

Неограниченость доказана.
\end{quote}
\end{quote}

\textbf{\textsf{(4)}}
\begin{quote}
    (a) 
    \begin{quote}
    $\forall \varepsilon > 0 \ \exists N \ \forall n > N: |\frac{n^2}{2n^2 + 3n + 1} - \frac{1}{2}| < \varepsilon$

    $|\frac{2n^2 - 2n^2 - 3n - 1}{4n^2 + 6n + 2}| = \frac{3 n + 1}{4n^2 + 6n + 2} < \frac{4n}{4n^2}$

    $\frac{1}{n} < \varepsilon \Leftrightarrow n > \frac{1}{\varepsilon}$
    
    Сходимость доказана. $(N = \left\lceil{\frac{1}{\varepsilon}}\right\rceil)$ 
    \end{quote}

    (b) 
    \begin{quote}
    $\forall \varepsilon > 0 \ \exists N \ \forall n > N: |\frac{\sin(n^2 + 2n - 1)}{n^3 - 3n + 2}| < \varepsilon$

    $|\frac{\sin(n^2 + 2n - 1)}{n^3 - 3n + 2}| < \frac{1}{n^3 - 3n + 2} < \frac{1}{n^3 - 3n} < \frac{n}{n^3 - 3n} = \frac{1}{n^2 - 3} < \varepsilon$

$
\begin{cases}
    n^2 - 3 > \frac{1}{\varepsilon} \\
    n > 2 \\
\end{cases}
$
    Сходимость доказана $(N = \max (3, \left\lceil{\sqrt{\frac{1}{\varepsilon} + 3}}\right\rceil))$.
    \end{quote}
\end{quote}

\textbf{\textsf{(5)}}
\begin{quote}
    
(a)
\begin{quote}
$
\lim\limits_{n \to \infty}(\frac{n^2 + 2n - 1}{2n - 7} - \frac{2n^2 + 1}{4n - 1}) = 
\lim\limits_{n \to \infty}\frac{n + 2 - \frac{1}{n}}{2 - \frac{7}{n}} - \lim\limits_{n \to \infty}\frac{2n + \frac{1}{n}}{4 - \frac{1}{n}} = \lim\limits_{n \to \infty}\frac{n + 2}{2} - \lim\limits_{n \to \infty}\frac{2n}{4} = \ =\lim\limits_{n \to \infty}\frac{n}{2} - \lim\limits_{n \to \infty}\frac{n}{2} + 1 = \boxed{1}
$
\end{quote}

(b)
\begin{quote}
$
\lim\limits_{n \to \infty}(\sqrt{n + \sqrt{n + \sqrt{n}}} - \sqrt{n}) = \lim\limits_{n \to \infty} \frac{\sqrt{n + \sqrt{n}}}{\sqrt{n + \sqrt{n + \sqrt{n}}} + \sqrt{n}} = \lim\limits_{n \to \infty}\frac{\sqrt {1 + \frac{\sqrt{n}}{n}}}{\sqrt{1 + \sqrt{\frac{1}{n} + \sqrt{\frac{n}{n^4}}}} + 1} = \ = \frac{\sqrt{1 + \lim \text{б.м.}}}{\sqrt{1 + \sqrt{\lim \text{б.м.} + \sqrt{\lim \text{б.м.}}}} + 1} = \boxed{\frac{1}{2}}
$
\end{quote}

(c)

\begin{quote}
\[
\lim \limits_{n \to \infty} \frac{\sqrt[n]{4n} + (1 + \frac{1}{2n})^n}{3 \sqrt[n]{4} - 2} = \lim \limits_{n \to \infty} \frac{\sqrt[n]{4}\sqrt[n]{n} + (1 + \frac{1}{2n})^n}{3 \sqrt[n]{4} - 2} = \frac{1 \cdot 1 + \lim\limits_{n \to \infty} (1 + \frac{1}{2n})^n}{3 \cdot 1 - 2} = 1 + \lim\limits_{t \to \infty} (1 + \frac{1}{t})^\frac{t}{2} = \boxed{1 + \sqrt{e}}
\]
\end{quote}
\end{quote}

\textbf{\textsf{6}}
\begin{quote}
    $a_n = \sqrt[n]{\frac{2n^3 - 7n + 3}{2n + 11}}$, для $n > 5$ числ. точно больше знаменателя, т.е. $a_{n > 5} > 1$
    
    $b_n = \{ -1, -1, -1, -1, -1, 1, 1, 1 ... \}$, предел $b_n = 1$

    $\sqrt[n]{\frac{2n^3 - 7n + 3}{2n + 11}} < \sqrt[n]{\frac{2n^3 + 3n}{2n}} = \sqrt[n]{\frac{2n^2 + 3}{2}} = \sqrt[n]{n^2 + \frac{3}{2}} < \sqrt[n]{n^2 + 2n} = \sqrt[n]{n} \sqrt[n]{n + 2} < \sqrt[n]{n} \sqrt[n]{3n} = \sqrt[n]{n} \sqrt[n]{n} \sqrt[n]{3} = c_n $, предел $c_n = 1$

    $b_n \leq a_n \leq c_n$ Тогда, по т. о зажатой послед., \boxed{\text{предел }$a_n = 1$}
\end{quote}
\end{document}