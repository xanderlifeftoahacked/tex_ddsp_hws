\documentclass{article}
\usepackage{ucs} 
\usepackage[utf8x]{inputenc} % Включаем поддержку UTF8  
\usepackage{amsmath}
\usepackage{tikz}
\usepackage{setspace}
\usepackage{amsfonts}
\usepackage{geometry}
\usepackage{quoting}
\usepackage[russian]{babel}  % Включаем пакет для поддержки русского языка  
\usepackage[a4paper, left=0.5cm, right=0.5cm, top=0cm, bottom=0cm]{geometry}
\begin{document}
\setlength{\parindent}{0pt}
\begin{Large}
    \textsf{\textbf{ЛинАл ДЗ №5}}
    
    Шамаев Александр    
\end{Large}
\vspace{1cm}

\textsf{\textbf{(1)}} (Виктор Евгеньевич разрешил не приводить к улучшенному ступ. виду, т.к. сразу видно, как выразить остальные вектора через базисные)

\begin{quote}
(a) Составим матрицу из векторов и приведем ее к транспонированному ступенчатому виду:

$\begin{pmatrix} 
\mathbf{v_1} & \mathbf{v_2} & \mathbf{v_3} & \mathbf{v_4} & \mathbf{v_5} \\
1 & 2 & 1 & 1 & 0 \\
0 & 1 & 1 & 2 & 1 \\
0 & 1 & 1 & 3 & 2 \\
-1 & 0 & 1 & 4 & 3 \\ 
\end{pmatrix}$ \Leftrightarrow
$\begin{pmatrix} 
\mathbf{v_1} & \mathbf{v_2} & \mathbf{v_3} & \mathbf{v_4} & \mathbf{v_5} \\
2 & 2 & 1 & 1 & 0 \\
1 & 1 & 1 & 2 & 1 \\
1 & 1 & 1 & 3 & 2 \\
0 & 0 & 1 & 4 & 3 \\ 
\end{pmatrix}$ \Leftrightarrow
$\begin{pmatrix} 
\mathbf{v_2} & \mathbf{v_3} & \mathbf{v_4} & \mathbf{v_5} & \mathbf{v_1}\\
2 & 1 & 1 & 0 & 0 \\
1 & 1 & 2 & 1 & 0 \\
1 & 1 & 3 & 2 & 0 \\
0 & 1 & 4 & 3 & 0 \\ 
\end{pmatrix}$ \Leftrightarrow

\hspace{0.8cm}
$\begin{pmatrix} 
\mathbf{v_2} & \mathbf{v_3} & \mathbf{v_4} & \mathbf{v_5} & \mathbf{v_1}\\
2 & 0 & 1 & 0 & 0 \\
1 & -1 & 2 & 1 & 0 \\
1 & -2 & 3 & 2 & 0 \\
0 & -3 & 4 & 3 & 0 \\ 
\end{pmatrix}$ \Leftrightarrow
$\begin{pmatrix} 
\mathbf{v_2} & \mathbf{v_3} & \mathbf{v_4} & \mathbf{v_5} & \mathbf{v_1}\\
2 & 0 & 0 & 0 & 0 \\
1 & -1 & \frac{3}{2} & 1 & 0 \\
1 & -2 & \frac{5}{2} & 2 & 0 \\
0 & -3 & 4 & 3 & 0 \\ 
\end{pmatrix}$ \Leftrightarrow
$\begin{pmatrix} 
\mathbf{v_2} & \mathbf{v_3} & \mathbf{v_4} & \mathbf{v_5} & \mathbf{v_1}\\
2 & 0 & 0 & 0 & 0 \\
1 & -1 & 0 & 1 & 0 \\
1 & -2 & 1 & 2 & 0 \\
0 & -3 & 4 & 3 & 0 \\ 
\end{pmatrix}$ \Leftrightarrow

\hspace{0.8cm}
$\begin{pmatrix} 
\mathbf{v_2} & \mathbf{v_3} & \mathbf{v_4} & \mathbf{v_5} & \mathbf{v_1}\\
2 & 0 & 0 & 0 & 0 \\
1 & -1 & 0 & 0 & 0 \\
1 & -2 & 1 & 0 & 0 \\
0 & -3 & 4 & 0 & 0 \\ 
\end{pmatrix}$
Получаем, что $v_2, v_3, v_4$ - линейно независимые

\hspace{0.8cm}
Размерность равна 3, базис могут составлять $v_2, v_3, v_4$, \quad $v_1 = v_2 - v_3$, \quad $v_5 = v_4 - v_3$ 

\hspace{0.8cm} ($v_1, v_2, v_3, v_4, v_5$ до элементарных преобразований)
\end{quote}

\begin{quote}
    (b) Составим матрицу из векторов и приведем ее к транспонированному ступенчатому виду:

    $
    \begin{pmatrix}
\mathbf{v_1} & \mathbf{v_2} & \mathbf{v_3} & \mathbf{v_4} & \mathbf{v_5} \\
    1 & 1 & 2 & 1 & 1 \\    
    1 & 1 & 2 & 1 & -1 \\    
    1 & -1 & 0 & 5 & -1 \\    
    1 & -1 & 0 & 5 & 0 \\    
    0 & -1 & -1 & 2 & 0 \\    
    \end{pmatrix} \Leftrightarrow
        \begin{pmatrix}
\mathbf{v_1} & \mathbf{v_2} & \mathbf{v_3} & \mathbf{v_4} & \mathbf{v_5} \\
    1 & 0 & 2 & 1 & 1 \\    
    1 & 0 & 2 & 1 & -1 \\    
    1 & -2 & 0 & 5 & -1 \\    
    1 & -2 & 0 & 5 & 0 \\    
    0 & -1 & -1 & 2 & 0 \\    
    \end{pmatrix} \Leftrightarrow
         \begin{pmatrix}
\mathbf{v_1} & \mathbf{v_2} & \mathbf{v_3} & \mathbf{v_4} & \mathbf{v_5} \\
    1 & 0 & 0 & 1 & 1 \\    
    1 & 0 & 0 & 1 & -1 \\    
    1 & -2 & -2 & 5 & -1 \\    
    1 & -2 & -2 & 5 & 0 \\    
    0 & -1 & -1 & 2 & 0 \\    
    \end{pmatrix} \Leftrightarrow
          \begin{pmatrix}
\mathbf{v_1} & \mathbf{v_5} & \mathbf{v_3} & \mathbf{v_4} & \mathbf{v_2} \\
    1 & 1 & 0 & 1 & 0 \\    
    1 & -1 & 0 & 1 & 0 \\    
    1 & -1 & -2 & 5 & 0 \\    
    1 & 0 & -2 & 5 & 0 \\    
    0 & 0 & -1 & 2 & 0 \\    
    \end{pmatrix} \Leftrightarrow
           \begin{pmatrix}
\mathbf{v_1} & \mathbf{v_5} & \mathbf{v_3} & \mathbf{v_4} & \mathbf{v_2} \\
    1 & 0 & 0 & 1 & 0 \\    
    1 & -2 & 0 & 1 & 0 \\    
    1 & -2 & -2 & 5 & 0 \\    
    1 & -1 & -2 & 5 & 0 \\    
    0 & 0 & -1 & 2 & 0 \\    
    \end{pmatrix} \Leftrightarrow
            \begin{pmatrix}
\mathbf{v_1} & \mathbf{v_5} & \mathbf{v_3} & \mathbf{v_4} & \mathbf{v_2} \\
    1 & 0 & 0 & 0 & 0 \\    
    1 & -2 & 0 & 0 & 0 \\    
    1 & -2 & -2 & 4 & 0 \\    
    1 & -1 & -2 & 4 & 0 \\    
    0 & 0 & -1 & 2 & 0 \\    
    \end{pmatrix} \Leftrightarrow
             \begin{pmatrix}
\mathbf{v_1} & \mathbf{v_5} & \mathbf{v_3} & \mathbf{v_4} & \mathbf{v_2} \\
    1 & 0 & 0 & 0 & 0 \\    
    1 & -2 & 0 & 0 & 0 \\    
    1 & -2 & -2 & 0 & 0 \\    
    1 & -1 & -2 & 0 & 0 \\    
    0 & 0 & -1 & 0 & 0 \\    
    \end{pmatrix}
    $ Получатеся, что $v_1, v_5, v_3$ - линейно независимые.

    Размерность равна 3, базис могут составлять $v_1, v_5, v_3$; \quad $v_2 = v_3 - v_1$, \ $v_4 = 5v_1 - 2v_3$ 
    
($v_1, v_2, v_3, v_4, v_5$ до элементарных преобразований)
\end{quote}

\textbf{\textsf{(2)}}
\begin{quote}
    (a) $
    \begin{pmatrix}
       2 & -4 & 5 & 3 \\ 
       3 & -6 & 4 & 2 \\ 
       4 & -8 & 17 & 11 \\ 
    \end{pmatrix} \Leftrightarrow
        \begin{pmatrix}
       2 & -4 & 5 & 3 \\ 
       3 & -6 & 4 & 2 \\ 
       0 & 0 & 7 & 5 \\ 
    \end{pmatrix} \Leftrightarrow
    \begin{pmatrix}
       2 & -4 & 5 & 3 \\ 
       0 & 0 & -\frac{7}{2} & -\frac{5}{2} \\ 
       0 & 0 & 7 & 5 \\ 
    \end{pmatrix} \Leftrightarrow
        \begin{pmatrix}
       1 & -2 & \frac{5}{2} & \frac{3}{2} \\ 
       0 & 0 & 7 & 5 \\ 
       0 & 0 & 0 & 0 \\ 
    \end{pmatrix} \Leftrightarrow
            \begin{pmatrix}
       1 & -2 & 0 & -\frac{2}{7} \\ 
       0 & 0 & 7 & 5 \\ 
       0 & 0 & 0 & 0 \\ 
    \end{pmatrix} \Leftrightarrow
                \begin{pmatrix}
       1 & -2 & 0 & -\frac{2}{7} \\ 
       0 & 0 & 1 & \frac{5}{7} \\ 
       0 & 0 & 0 & 0 \\ 
    \end{pmatrix} 
    $
    
    Общее решение: $\begin{cases} x_3 = -\frac{5}{7} x_4 \\ x_1 = 2x_2 + \frac{2}{7} x_4\end{cases}$ $X = \begin{pmatrix}2x_2 + \frac{2}{7}x_4 \\ 
                          x_2 \\ -\frac{5}{7} x_4 \\ x_4 \end{pmatrix}$

    ФСР = $\{(2, 1, 0 , 0)^T , (\frac{2}{7}, 0, -\frac{5}{7}, 1)^T \}$
\end{quote}

\begin{quote}
    (b) 
$
\begin{pmatrix}
3 & 5 & 2 \\        
4 & 7 & 5 \\        
1 & 1 & -4 \\        
2 & 9 & 6 \\        
\end{pmatrix} \Leftrightarrow
\begin{pmatrix}
3 & 5 & 2 \\        
4 & 7 & 5 \\        
1 & 1 & -4 \\        
0 & 7 & 14 \\        
\end{pmat2rix} \Leftrightarrow
\begin{pmatrix}
3 & 5 & 2 \\        
4 & 7 & 5 \\        
0 & -\frac{3}{4} & -\frac{21}{4} \\        
0 & 7 & 14 \\        
\end{pmatrix} \Leftrightarrow
\begin{pmatrix}
3 & 5 & 2 \\        
0 & \frac{1}{3} & \frac{7}{3} \\        
0 & -\frac{3}{4} & -\frac{21}{4} \\        
0 & 7 & 14 \\        
\end{pmatrix} \Leftrightarrow
\begin{pmatrix}
3 & 5 & 2 \\        
0 & \frac{1}{3} & \frac{7}{3} \\        
0 & 7 & 14 \\        
0 & 0 & 0 \\        
\end{pmatrix} \Leftrightarrow
\begin{pmatrix}
1 & \frac{5}{3} & \frac{2}{3} \\        
0 & 1 & 7 \\        
0 & 7 & 14 \\        
0 & 0 & 0 \\        
\end{pmatrix} \Leftrightarrow
\begin{pmatrix}
1 & \frac{5}{3} & \frac{2}{3} \\        
0 & 7 & 14 \\        
0 & 1 & 7 \\        
0 & 0 & 0 \\        
\end{pmatrix} \Leftrightarrow
\begin{pmatrix}
1 & \frac{5}{3} & \frac{2}{3} \\        
0 & 7 & 14 \\        
0 & 0 & 5 \\        
0 & 0 & 0 \\        
\end{pmatrix} \Leftrightarrow
\begin{pmatrix}
1 & 0 & -\frac{8}{3} \\        
0 & 7 & 14 \\        
0 & 0 & 5 \\        
0 & 0 & 0 \\        
\end{pmatrix} \Leftrightarrow
\begin{pmatrix}
1 & 0 & 0 \\        
0 & 1 & 0 \\        
0 & 0 & 5 \\        
0 & 0 & 0 \\        
\end{pmatrix}
$

Общее решение: $X = \begin{pmatrix}
    0 \\ 0 \\ 0
\end{pmatrix}$

ФСР = $\{ (0, 0, 0)^T\}$

(c) 
$
\begin{pmatrix}
1 & 0 & -1 & 0 & 1 & 0 \\    
0 & 1 & 0 & -1 & 0 & 1 \\    
1 & -1 & 0 & 0 & 1 & -1\\    
0 & 1 & -1 & 0 & 0  & 1\\    
1 & 0 & 0 & -1 & 1 & 0 \\    
\end{pmatrix} \Leftrightarrow
\begin{pmatrix}
1 & 0 & -1 & 0 & 1 & 0 \\    
0 & 1 & 0 & -1 & 0 & 1 \\    
1 & -1 & 0 & 0 & 1 & -1\\    
0 & 1 & -1 & 0 & 0  & 1\\    
0 & 0 & 1 & -1 & 0 & 0 \\    
\end{pmatrix} \Leftrightarrow
\begin{pmatrix}
1 & 0 & -1 & 0 & 1 & 0 \\    
0 & 1 & 0 & -1 & 0 & 1 \\    
0 & -1 & 1 & 0 & 0 & -1\\    
0 & 1 & -1 & 0 & 0  & 1\\    
0 & 0 & 1 & -1 & 0 & 0 \\    
\end{pmatrix} \Leftrightarrow
\begin{pmatrix}
1 & 0 & -1 & 0 & 1 & 0 \\    
0 & 1 & 0 & -1 & 0 & 1 \\    
0 & -1 & 1 & 0 & 0 & -1\\    
0 & 0 & -1 & 1 & 0  & 0\\    
0 & 0 & 1 & -1 & 0 & 0 \\    
\end{pmatrix} \Leftrightarrow
\begin{pmatrix}
1 & 0 & -1 & 0 & 1 & 0 \\    
0 & 1 & 0 & -1 & 0 & 1 \\    
0 & 0 & 1 & -1 & 0 & 0\\    
0 & 0 & -1 & 1 & 0  & 0\\    
0 & 0 & 1 & -1 & 0 & 0 \\    
\end{pmatrix} \Leftrightarrow
\begin{pmatrix}
1 & 0 & -1 & 0 & 1 & 0 \\    
0 & 1 & 0 & -1 & 0 & 1 \\    
0 & 0 & 1 & -1 & 0 & 0\\    
0 & 0 & -1 & 1 & 0  & 0\\    
0 & 0 & 0 & 0 & 0 & 0 \\    
\end{pmatrix} \Leftrightarrow
\begin{pmatrix}
1 & 0 & -1 & 0 & 1 & 0 \\    
0 & 1 & 0 & -1 & 0 & 1 \\    
0 & 0 & 1 & -1 & 0 & 0\\    
0 & 0 & 0 & 0 & 0  & 0\\    
0 & 0 & 0 & 0 & 0 & 0 \\    
\end{pmatrix} \Leftrightarrow
\begin{pmatrix}
1 & 0 & 0 & -1 & 1 & 0 \\    
0 & 1 & 0 & -1 & 0 & 1 \\    
0 & 0 & 1 & -1 & 0 & 0\\    
0 & 0 & 0 & 0 & 0  & 0\\    
0 & 0 & 0 & 0 & 0 & 0 \\    
\end{pmatrix} 
$

Общее решение: $X = \begin{pmatrix}
    x_4 - x_5 \\ 
    x_4 - x_6 \\
    x_4 \\
    x_5 \\ 
    x_6 \\
\end{pmatrix}$

ФСР: $ \{(1, 1, 1, 0, 0)^T, (-1, 0, 0, 1, 0)^T, (0, -1, 0, 0, 0, 1)^T \} $

\end{quote}

\end{document}